\documentclass{article}
\usepackage{ae,aecompl}
\usepackage{todonotes}
\usepackage{chngcntr}
\usepackage{tikz-cd}
\usepackage{graphicx}
\graphicspath{ {./images/}}
\usepackage[all,cmtip]{xy}
\usepackage{amsmath, amscd}
\usepackage{amsthm}
\usepackage{amssymb}
\usepackage{amsfonts}
\usepackage{bm}
\usepackage{qsymbols}
\usepackage{latexsym}
\usepackage{mathrsfs}
\usepackage{mathtools}
\usepackage{cite}
\usepackage{color}
\usepackage{url}
\usepackage{enumerate}
\usepackage{verbatim}
\usepackage[draft=false, colorlinks=true]{hyperref}
\usepackage{pdfpages}
\usepackage[margin=1.2in]{geometry}
\usepackage{IEEEtrantools}
\usepackage{multirow}
\usepackage{fancyhdr}


\usepackage[nameinlink]{cleveref}


\DeclareMathOperator*{\ac}{accept}
\DeclareMathOperator*{\amax}{argmax}
\DeclareMathOperator*{\amin}{argmin}
\DeclareMathOperator*{\Aut}{Aut}
\newcommand {\al}{{\alpha}}
\newcommand {\abs}[1]{{\left\lvert#1\right\rvert}}
\newcommand {\A}{{\mathcal{A}}}
\newcommand {\AM}{{\mathrm{AM}}}
\newcommand {\AMp}{{\AM_{p}^{X}\!(\Ri_\w)}}
\newcommand {\B}{{\mathcal{B}}}
\DeclareMathOperator*{\Be}{Bern}
\newcommand {\Br}{{\dot{B}}}
\newcommand {\Ba}{{\mathfrak{B}}}
\newcommand {\C}{{\mathbb C}}
\newcommand {\ce}{\mathrm{c}}
\newcommand {\Ce}{\mathrm{C}}
\newcommand {\Cc}{\mathrm{C_{c}}}
\newcommand {\Ccinf}{\mathrm{C_{c}^{\infty}}}
\DeclareMathOperator{\DEV}{DEV}
\DeclareMathOperator{\diag}{diag}
\newcommand {\Di}{{\mathbb D}}
\newcommand {\dom}{\mathrm{dom}}
\newcommand {\ud}{\mathrm{d}}
\newcommand {\ue}{\mathrm{e}}
\newcommand {\eps}{\varepsilon}
\newcommand {\veps}{\varepsilon}
\newcommand {\vrho}{{\varrho}}
\newcommand {\E}{{\mathbb{E}}}
\newcommand {\Ec}{{\mathcal{E}}}
\newcommand {\Ell}{L}
\newcommand {\Ellp}{{L_{p}[0,1]}}
\newcommand {\Ellpprime}{{L_{p'}([0,1])}}
\newcommand {\Ellq}{{L_{q}([0,1])}}
\newcommand {\Ellqprime}{{L_{q'}([0,1])}}
\newcommand {\Ellr}{L^{r}}
\newcommand {\Ellone}{{L_{1}([0,1])}}
\newcommand {\Elltwo}{{L_{2}([0,1])}}
\newcommand {\Ellinfty}{L^{\infty}}
\newcommand {\Ellinftyc}{L_{\mathrm{c}}^{\infty}}
\newcommand {\exb}[1]{\exp\left\{#1\right\}}
\DeclareMathOperator*{\Exp}{Exp}
\DeclareMathOperator*{\Ext}{Ext}
\newcommand {\F}{{\mathcal{F}}}
\newcommand{\Fe}{{\mathbb{F}}}
\newcommand
{\G}{{\mathcal{G}}}
\newcommand {\HF}{\mathcal{H}_{\text{FIO}}^{1}(\Rd)}
\newcommand {\Hr}{H}
\newcommand {\HT}{\mathcal{H}}
\newcommand {\ui}{\mathrm{i}}
\newcommand {\I}{{I}}
\newcommand {\J}{{\mathcal{J}}}
\newcommand {\id}{{\mathrm{id}}}
\newcommand {\iid}{\stackrel{\mathclap{\normalfont\mbox{iid}}}{\sim}}
\newcommand {\im}{{\text{im }}}
\newcommand {\ind}{{\perp\!\!\!\perp}}
\newcommand{\indep}{\stackrel{\text{Indep}}{\sim}}
\DeclareMathOperator*{\Int}{int}
\newcommand {\intx}{{\overline{\int_{X}}}}
\newcommand {\inte}{{\overline{\int_{\E}}}}
\newcommand {\la}{\lambda}
\newcommand {\rb}{\rangle}
\newcommand {\lb}{{\langle}}
\newcommand {\La}{\Lambda}
\newcommand {\calL}{{\mathcal{L}}}
\newcommand {\lp}{{\mathcal{L}}^{p}}
\newcommand {\lpo}{{\overline{\mathcal{L}}^{p}\!}}
\newcommand {\Lpo}{{\overline{\Ell}^{p}\!}}
\newcommand {\M}{{\mathbf{M}}}
\newcommand {\Ma}{{\mathcal{M}}}
\newcommand {\N}{{{\mathbb N}}}
\newcommand {\Na}{{{\mathcal{N}}}}
\newcommand {\norm}[1]{\left\|#1\right\|}
\newcommand {\normm}[1]{{\left\vert\kern-0.25ex\left\vert\kern-0.25ex\left\vert #1 
    \right\vert\kern-0.25ex\right\vert\kern-0.25ex\right\vert}}
\newcommand {\Om}{{{\Omega}}}
\newcommand {\one}{{{\bf 1}}}
\newcommand {\pic}{\text{Pic }}
\newcommand {\ph}{{\varphi}}
\newcommand {\Pa}{{\mathbb{P}}}
\newcommand {\Po}{{\mathcal{P}}}
\newcommand {\Q}{{\mathbb{Q}}}
\newcommand {\R}{{\mathbb R}}
\newcommand {\Rd}{{\mathbb{R}^{d}}}
\DeclareMathOperator{\rej}{reject }
\newcommand {\Rn}{{\mathbb{R}^{n}}}
\newcommand {\cR}{{\mathcal{R}}}
\newcommand {\Rad}{{\mathrm{Rad}}}
\newcommand {\ran}{{\mathrm{ran}}}
\newcommand {\Ri}{{\mathrm{R}}}
\newcommand {\supp}{{\mathrm{supp}}}
\newcommand {\Se}{\mathrm{S}}
\newcommand {\Sp}{S^{*}(\Rn)}
\newcommand {\St}{{\mathrm{St}}}
\newcommand {\Sw}{\mathcal{S}}
\newcommand {\T}{{\mathcal{T}}}
\newcommand {\ta}{{\theta}}
\newcommand {\Ta}{{\Theta}}
\DeclareMathOperator {\V}{Var}
\newcommand {\w}{{\omega}}
\newcommand {\W}{{\mathrm{W}}}
\newcommand {\Wnp}{\text{$\mathrm{W}$\textsuperscript{$n,\!p$}}}
\newcommand {\Wnpeq}{\text{$\mathrm{W}$\textsuperscript{$n\!,\!p$}}}
\newcommand {\Wonep}{\text{$\mathrm{W}$\textsuperscript{$1,\!p$}}}
\newcommand {\Wonepeq}{\text{$\mathrm{W}$\textsuperscript{$1\!,\!p$}}}
\newcommand {\X}{{\mathcal{X}}}
\newcommand {\Z}{{{\mathbb Z}}}
\newcommand {\Za}{{\mathcal{Z}}}
\newcommand {\Zd}{{\Z[\sqrt{d}]}}
\newcommand {\vanish}[1]{\relax}

\newcommand {\wh}{\widehat}
\newcommand {\wt}{\widetilde}
\newcommand {\red}{\color{red}}

% Distributions
\newcommand{\normal}{\mathsf{N}}
\newcommand{\poi}{\mathsf{Poisson}}
\newcommand{\bern}{\mathsf{Bernoulli}}
\newcommand{\bin}{\mathsf{Binomal}}
\newcommand{\multi}{\mathsf{Multinomial}}




% put your command and environment definitions here




% some theorem environments
% remove "[theorem]" if you do not want them to use the same number sequence


  \newtheorem{thrm}{Theorem}
  \newtheorem{lemma}{Lemma}
  \newtheorem{prop}{Proposition}
  \newtheorem{cor}{Corollary}

  \newtheorem{conj}{Conjecture}
  \renewcommand{\theconj}{\Alph{conj}}  % numbered A, B, C etc

  \theoremstyle{definition}
  \newtheorem{defn}{Definition}
  \newtheorem{ex}{Example}
  \newtheorem{exs}{Examples}
  \newtheorem{question}{Question}
  \newtheorem{remark}{Remark}
  \newtheorem{notn}{Notation}
  \newtheorem{exer}{Exercise}




\title{STATS300A - Lecture 15}
\author{Dominik Rothenhaeusler\\ Scribed by Michael Howes}
\date{11/10/21}

\pagestyle{fancy}
\fancyhf{}
\rhead{STATS300A - Lecture 15}
\lhead{11/10/21}
\rfoot{Page \thepage}

\begin{document}
\maketitle
\tableofcontents
\section{Motivation}
Today we are going to develop optimal tests in one-dimensional exponential families. A UMP test need not exist in this case. Indeed, in the Gaussian model there is no UMP test for $H_0 : \mu =\mu_0$ against $H_1 : \mu \neq \mu_0$. This is because for an alternative $\mu_1 < \mu_0$, the MP test rejects $H_0$ when our data is small. For $\mu_1 >\mu_0$ the MP test rejects $H_0$ when our data is large (see picture below).
\begin{center}
    
\end{center}


There are various approaches that can be used when the UMP test does not exist. These include:
\begin{enumerate}
    \item Symmetric constraints on test. For example $\phi (c x) = \phi(x)$ for $c >0$.
    \item Put a probability distribution on $\Om_1$. Then collapse the risk function and maximize the average power.
    \item Maximize the worst case power (maximin test).
    \item Monotonicity constraints.
    \item Unbiasedness.
\end{enumerate}
\section{Unbiased tests}
We will use the unbiasedness approach but it is good to know that there are other method available. 
\begin{defn}
    A test $\phi$ is \emph{unbiased} at level $\al$ if
    \begin{enumerate}
        \item For all $\ta_0 \in \Om_0$, $\E_{\ta_0} \phi \le \al$.
        \item For all $\ta_1 \in \Om_1$, $\E_{\ta_1} \phi \ge \al$.
    \end{enumerate}
\end{defn}
\begin{remark}
    Some remarks on the above definition
    \begin{itemize}
        \item This is a form of risk unbiasedness. It corresponds to having level $\al$ and being risk unbiased under 0-1 loss.
        \item UMP tests (if they exist) are always unbiased tests.
        \item Recommending reading: \href{https://arxiv.org/abs/1410.2597}{Optimal inference after model selection} by Fiftian, Sun and Taylor. This is a recent paper that uses ideas of unbiased tests.
    \end{itemize}
\end{remark}
The definition of unbiased gives us a new optimality condition. 
\begin{defn}
    A \emph{uniformly most power unbiased (UMPU) test} is an unbiased level $\al$ test which is uniformly most powerful among all unbiased level $\al$ tests.
\end{defn}
Thus we are restricting our set of tests. We have 
\[\text{unbiased test} \subsetneq \text{all tests}. \]
A UMPU test is a test in the first class which is uniformly optimal in the first class. Unlike UMP tests, UMPU tests often exist for test $H_0 : \ta =\ta_0$ against $H_1 : \ta \neq \ta_0$ and for testing $H_0 : \ta \le \ta_0$ against $H_1 : \ta > \ta_0$ in multiple parameter settings.
\section{Constructing UMPU tests}
\subsection{A recipe}
Constructing UMPU tests is not a straight forward proceedure. The general recipe that we will use is
\begin{enumerate}
    \item Rewrite constraints as weaker constraints. We will replace unbiasedness with a weaker condition.
    \item Fix a simple alternative $\ta_1 \in \Om_1$.
    \item Find a MP under the constraints from (a). This will require a generalized Neyman Pearson lemma. Call this test $\phi_{\ta_1}$.
    \item If the test $\phi = \phi_{\ta_1}$ does not depend on $\ta_1$, then $\phi$ is in fact the UMP under the constraints in (a).
    \item Show that $\phi$ actuallly satisfies the stronger unbiased constraint and thus $\phi$ is UMPU.
\end{enumerate}
We will now work on each of the ingredients.
\subsection{$\al$-similar tests}
Typically $\Om_0$ and $\Om_1$ are subsets of $\R^k$ and so we can define 
\[W = \overline{\Om}_0 \cap \overline{\Om}_1, \]
where $\overline{A}$ is used to denote the closure of $A \subseteq \R^k$. 
\begin{exs}
\begin{itemize}
    \item If we have the test $H_0: \ta = \ta_0$ and $H_1 : \ta \neq \ta_0$, then $W = \{\ta_0\}$.
    \item If we have the test $H_0 : \ta_1 \le \wt{\ta}$ and $H_1 \ta_1 > \wh{\ta}$ where $\ta = (\ta_1,\ldots, \ta_k) \in \R^k$, then \[W = \{\ta \in \R^k : \ta_1 = \wt{\ta}\}.\]
\end{itemize}
\end{exs}
\begin{defn}
    A test $\phi$ is \emph{$\al$-similar} if $\E_\ta \phi = \al$ for all $\ta \in W$.
\end{defn}
\begin{lemma}
    Suppose that the function $\beta_\phi(\ta) = \E_\ta \phi$ is continuous on $\Om$ for all test functions $\phi$. If a test $\phi_0$ is UMP among all $\al$-similar tests, then $\phi_0$ is also UMP among all unbiased test.
\end{lemma}
\begin{proof}
    We will first show that $\phi_0$ is unbiased. Consider the constant test $\phi = \al$. The test $\phi$ is $\al$-similar. Since $\phi_0$ is UMP among all $\al$-similar tests we must have that for all $\ta \in \Om_1$,
    \[\E_{\ta_1}\phi_0 \ge \E_{\ta_1} \phi = \al. \]
    Thus $\phi_0$ is unbiased at level $\al$. To show that $\phi_0$ is the UMPU test, it suffices to show that every unbiased test is a level $\al$ test. If $\phi$ is unbiased, then by continuity of $\beta_\phi$ we have that 
    \[\E_{\ta} \phi = \beta_\phi(\ta) \le \al,~~~\text{for all } \ta \in \overline{\Om}_0. \]
    Likewise,
    \[\E_{\ta} \phi = \beta_\phi(\ta) \ge \al,~~~\text{for all } \ta \in \overline{\Om}_1. \]
    Thus $\E_\ta \phi = \al$ for all $\ta \in W = \overline{\Om}_0 \cap \overline{\Om}_1$.
\end{proof}
The idea behind this proof is that we have shown:

Thus a test with UMP among $\al$-similar tests is UMP among unbiased tests.

\subsection{Method of undetermined multipliers}
We will now restrict our attention to one-dimensional exponential families so that 
\[p_\ta(x) \propto h(a)\exb{\ta T(x)}.\]
We wish to test $H_0 : \ta=\ta_0$ against $H_1 :\ta \neq \ta_0$. Recall that no UMP test exists. One can show by the dominated convergence theorem that the map $\beta_\phi(\ta) = \E_\ta \phi$ is continuous and in fact differentiable for all test functions $\phi$. 

By our lemma from before we know that if $\phi$ is unbiased then $\E_{\ta_0}\phi = \al$ and $\E_\ta \phi \ge \al$ for all $\ta$. Thus every unbiased level $\al$ test must statisfy $\E_{\ta_0}\phi=\al$ and $\ta_0$ must be a global minimum of $\ta \mapsto \beta_\phi(\ta)$. So our power function looks something like this:



Thus the derivative of the power function at $\ta=\ta_0$ must be zerp. Hence every unbiased test must satisfy the two constraints:
\begin{equation}
    \int_\X \phi(x) p_\ta(x)d\mu(x) = \al ~~~\text{and}~~~ \partial_\ta \beta_\phi(\ta)\mid_{\ta = \ta_0} = \int_\X \phi(x) \partial_\ta p_{\ta_0}(x)d\mu(x) =0.
\end{equation}
In the second constrain we used some of the regularity properties of exponential families to exchange integration and differentiation.

\end{document}