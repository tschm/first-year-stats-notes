\documentclass{article}
\usepackage{ae,aecompl}
\usepackage{todonotes}
\usepackage{chngcntr}
\usepackage{tikz-cd}
\usepackage{graphicx}
\graphicspath{ {./images/}}
\usepackage[all,cmtip]{xy}
\usepackage{amsmath, amscd}
\usepackage{amsthm}
\usepackage{amssymb}
\usepackage{amsfonts}
\usepackage{bm}
\usepackage{qsymbols}
\usepackage{latexsym}
\usepackage{mathrsfs}
\usepackage{mathtools}
\usepackage{cite}
\usepackage{color}
\usepackage{url}
\usepackage{enumerate}
\usepackage{verbatim}
\usepackage[draft=false, colorlinks=true]{hyperref}
\usepackage{pdfpages}
\usepackage[margin=1.2in]{geometry}
\usepackage{IEEEtrantools}
\usepackage{multirow}
\usepackage{fancyhdr}


\usepackage[nameinlink]{cleveref}


\DeclareMathOperator*{\ac}{accept}
\DeclareMathOperator*{\amax}{argmax}
\DeclareMathOperator*{\amin}{argmin}
\DeclareMathOperator*{\Aut}{Aut}
\newcommand {\al}{{\alpha}}
\newcommand {\abs}[1]{{\left\lvert#1\right\rvert}}
\newcommand {\A}{{\mathcal{A}}}
\newcommand {\AM}{{\mathrm{AM}}}
\newcommand {\AMp}{{\AM_{p}^{X}\!(\Ri_\w)}}
\newcommand {\B}{{\mathcal{B}}}
\DeclareMathOperator*{\Be}{Bern}
\newcommand {\Br}{{\dot{B}}}
\newcommand {\Ba}{{\mathfrak{B}}}
\newcommand {\C}{{\mathbb C}}
\newcommand {\ce}{\mathrm{c}}
\newcommand {\Ce}{\mathrm{C}}
\newcommand {\Cc}{\mathrm{C_{c}}}
\newcommand {\Ccinf}{\mathrm{C_{c}^{\infty}}}
\DeclareMathOperator{\DEV}{DEV}
\DeclareMathOperator{\diag}{diag}
\newcommand {\Di}{{\mathbb D}}
\newcommand {\dom}{\mathrm{dom}}
\newcommand {\ud}{\mathrm{d}}
\newcommand {\ue}{\mathrm{e}}
\newcommand {\eps}{\varepsilon}
\newcommand {\veps}{\varepsilon}
\newcommand {\vrho}{{\varrho}}
\newcommand {\E}{{\mathbb{E}}}
\newcommand {\Ec}{{\mathcal{E}}}
\newcommand {\Ell}{L}
\newcommand {\Ellp}{{L_{p}[0,1]}}
\newcommand {\Ellpprime}{{L_{p'}([0,1])}}
\newcommand {\Ellq}{{L_{q}([0,1])}}
\newcommand {\Ellqprime}{{L_{q'}([0,1])}}
\newcommand {\Ellr}{L^{r}}
\newcommand {\Ellone}{{L_{1}([0,1])}}
\newcommand {\Elltwo}{{L_{2}([0,1])}}
\newcommand {\Ellinfty}{L^{\infty}}
\newcommand {\Ellinftyc}{L_{\mathrm{c}}^{\infty}}
\newcommand {\exb}[1]{\exp\left\{#1\right\}}
\DeclareMathOperator*{\Exp}{Exp}
\DeclareMathOperator*{\Ext}{Ext}
\newcommand {\F}{{\mathcal{F}}}
\newcommand{\Fe}{{\mathbb{F}}}
\newcommand
{\G}{{\mathcal{G}}}
\newcommand {\HF}{\mathcal{H}_{\text{FIO}}^{1}(\Rd)}
\newcommand {\Hr}{H}
\newcommand {\HT}{\mathcal{H}}
\newcommand {\ui}{\mathrm{i}}
\newcommand {\I}{{I}}
\newcommand {\J}{{\mathcal{J}}}
\newcommand {\id}{{\mathrm{id}}}
\newcommand {\iid}{\stackrel{\mathclap{\normalfont\mbox{iid}}}{\sim}}
\newcommand {\im}{{\text{im }}}
\newcommand {\ind}{{\perp\!\!\!\perp}}
\newcommand{\indep}{\stackrel{\text{Indep}}{\sim}}
\DeclareMathOperator*{\Int}{int}
\newcommand {\intx}{{\overline{\int_{X}}}}
\newcommand {\inte}{{\overline{\int_{\E}}}}
\newcommand {\la}{\lambda}
\newcommand {\rb}{\rangle}
\newcommand {\lb}{{\langle}}
\newcommand {\La}{\Lambda}
\newcommand {\calL}{{\mathcal{L}}}
\newcommand {\lp}{{\mathcal{L}}^{p}}
\newcommand {\lpo}{{\overline{\mathcal{L}}^{p}\!}}
\newcommand {\Lpo}{{\overline{\Ell}^{p}\!}}
\newcommand {\M}{{\mathbf{M}}}
\newcommand {\Ma}{{\mathcal{M}}}
\newcommand {\N}{{{\mathbb N}}}
\newcommand {\Na}{{{\mathcal{N}}}}
\newcommand {\norm}[1]{\left\|#1\right\|}
\newcommand {\normm}[1]{{\left\vert\kern-0.25ex\left\vert\kern-0.25ex\left\vert #1 
    \right\vert\kern-0.25ex\right\vert\kern-0.25ex\right\vert}}
\newcommand {\Om}{{{\Omega}}}
\newcommand {\one}{{{\bf 1}}}
\newcommand {\pic}{\text{Pic }}
\newcommand {\ph}{{\varphi}}
\newcommand {\Pa}{{\mathbb{P}}}
\newcommand {\Po}{{\mathcal{P}}}
\newcommand {\Q}{{\mathbb{Q}}}
\newcommand {\R}{{\mathbb R}}
\newcommand {\Rd}{{\mathbb{R}^{d}}}
\DeclareMathOperator{\rej}{reject }
\newcommand {\Rn}{{\mathbb{R}^{n}}}
\newcommand {\cR}{{\mathcal{R}}}
\newcommand {\Rad}{{\mathrm{Rad}}}
\newcommand {\ran}{{\mathrm{ran}}}
\newcommand {\Ri}{{\mathrm{R}}}
\newcommand {\supp}{{\mathrm{supp}}}
\newcommand {\Se}{\mathrm{S}}
\newcommand {\Sp}{S^{*}(\Rn)}
\newcommand {\St}{{\mathrm{St}}}
\newcommand {\Sw}{\mathcal{S}}
\newcommand {\T}{{\mathcal{T}}}
\newcommand {\ta}{{\theta}}
\newcommand {\Ta}{{\Theta}}
\DeclareMathOperator {\V}{Var}
\newcommand {\w}{{\omega}}
\newcommand {\W}{{\mathrm{W}}}
\newcommand {\Wnp}{\text{$\mathrm{W}$\textsuperscript{$n,\!p$}}}
\newcommand {\Wnpeq}{\text{$\mathrm{W}$\textsuperscript{$n\!,\!p$}}}
\newcommand {\Wonep}{\text{$\mathrm{W}$\textsuperscript{$1,\!p$}}}
\newcommand {\Wonepeq}{\text{$\mathrm{W}$\textsuperscript{$1\!,\!p$}}}
\newcommand {\X}{{\mathcal{X}}}
\newcommand {\Z}{{{\mathbb Z}}}
\newcommand {\Za}{{\mathcal{Z}}}
\newcommand {\Zd}{{\Z[\sqrt{d}]}}
\newcommand {\vanish}[1]{\relax}

\newcommand {\wh}{\widehat}
\newcommand {\wt}{\widetilde}
\newcommand {\red}{\color{red}}

% Distributions
\newcommand{\normal}{\mathsf{N}}
\newcommand{\poi}{\mathsf{Poisson}}
\newcommand{\bern}{\mathsf{Bernoulli}}
\newcommand{\bin}{\mathsf{Binomal}}
\newcommand{\multi}{\mathsf{Multinomial}}




% put your command and environment definitions here




% some theorem environments
% remove "[theorem]" if you do not want them to use the same number sequence


  \newtheorem{thrm}{Theorem}
  \newtheorem{lemma}{Lemma}
  \newtheorem{prop}{Proposition}
  \newtheorem{cor}{Corollary}

  \newtheorem{conj}{Conjecture}
  \renewcommand{\theconj}{\Alph{conj}}  % numbered A, B, C etc

  \theoremstyle{definition}
  \newtheorem{defn}{Definition}
  \newtheorem{ex}{Example}
  \newtheorem{exs}{Examples}
  \newtheorem{question}{Question}
  \newtheorem{remark}{Remark}
  \newtheorem{notn}{Notation}
  \newtheorem{exer}{Exercise}




\title{STATS300A - Lecture 9}
\author{Dominik Rothenhaeusler\\ Scribed by Michael Howes}
\date{10/18/21}

\pagestyle{fancy}
\fancyhf{}
\rhead{STATS300A - Lecture 9}
\lhead{10/18/21}
\rfoot{Page \thepage}

\begin{document}
\maketitle
\tableofcontents
\section{Overview}
We have been studying optimal estimation. We have tried multiple things:
\begin{enumerate}
    \item Uniform comparisions.
    \item Restricting the class of estimtors.
    \item Collapsing the risk.\begin{enumerate}
        \item Bayesian estimators.
        \item Minimax estimators.
    \end{enumerate}
\end{enumerate}
Today we will discuss:
\begin{itemize}
    \item Some strengths of Bayesian techniques.
    \item Minimax risk estimation.
\end{itemize}
\section{Stengths of the Bayesian approach}
\begin{lemma}
    \emph{[TPE 4.14]} Let $Q$ be the marginal distribution of $X$. That is
    \[Q(A) = \int_\Om \Pa_\ta(X\in A)d\Lambda. \]
    If the loss is strictly convex in $d$, then the Bayes estimator $\delta_\Lambda$ is unique a.s. $\Pa_\ta$ for all $\ta \in \Om$ if
    \begin{enumerate}
        \item $r(\Lambda, d_\Lambda) < \infty$.
        \item If $A \subseteq \X$ and $Q(A) = 0$, then $\Pa_\ta(A) = 0$ for all $\ta \in \Om$.
    \end{enumerate}
\end{lemma}
See the textbook for a proof. Note that if the following all hold, then we can conclude that (b) holds above.
\begin{enumerate}
    \item $\Om$ is an open subset of $\R^k$.
    \item The map $\ta \to \Pa_\ta(A)$ is continuous for all $A$.
    \item $\pi(\ta) > 0$ for all $\ta \in \Om$.
    \item $\ta \to \pi(\ta)$ is continuous.
\end{enumerate}
\subsection{Bayesian recursion}
Suppose $\ta \sim \Lambda$ and $X_i \iid \Pa_\ta$. We can then update the prior sequentially. Note that the posterior for $m <n$ observations is
\[p(\ta|X_1,\ldots,X_n) \propto \text{likelihood}\times\text{prior} = p(X_1,\ldots, X_m|\ta)\pi(\ta). \]
The posterior for the full sample of $n$ observations is
\begin{align*}
    p(\ta|X_1,\ldots,X_n)&\propto \text{likelihood}\times \text{prior}\\
    &= p(X_1,\ldots,X_n | \ta) \pi(\ta)\\
    &= p(X_1,\ldots, X_{m}|\ta) p(X_{m+1},\ldots, X_n | \ta) \pi(\ta)\\
    &\propto p(X_{m+1},\ldots, X_n | \ta)p(\ta|X_1,\ldots, X_m).
\end{align*} 
Thus we can think of $X_{m+1},\ldots,X_n$ as new data and the posterior $p(\ta|X_1,\ldots, X_m)$ as a new prior. It follows that we can compute posteriors recursively by changing the prior. This allows for speedy calculations.
\begin{ex}
    Suppose $X_1,\ldots,X_n\iid \Na(\ta,\sigma^2)$ where $\sigma^2$ is known and $\ta \sim \Na(\mu,b^2)$. We saw that 
    \[\text{posterior} \propto \exb{-\frac{1}{2}\ta^2 w_n+\ta \bar{w}_n}, \]
    where
    \begin{align*}
        w_n &= \frac{n}{\sigma^2}+\frac{1}{b^2},\\
        \bar{w}_n &= \frac{\sum_{i=1}^n x_i}{\sigma^2}+\frac{\mu}{b^2}.
    \end{align*}
    The weights $w_n$ satisfy the recursion
    \begin{align*}
        w_n &= \frac{1}{\sigma^2}+w_{n-1},\\
        \bar{w}_n &= \frac{x_n}{\sigma^2}+\bar{w}_{n-1}.
    \end{align*}
    Thus our updates are quick linear calculations. This has many applications.
\end{ex}
\subsection{Hierarchical models and empirical Bayes}
We can use Bayesian ideas to model problems with repeat structure and pool information across observations. Suppose we have $\ta_i \iid \Na(0,\tau^2)$ for $i=1,\ldots,p$ where $\tau^2 > 0$ is known (for now). Suppose that we also have $X_i \stackrel{\text{ind}}{\sim} \Na(\ta_i,1)$. For example $\ta_i$ might be the effect of an experiment and $X_i$ is the measured effect and our measurement errors are i.i.d. $\Na(0,1)$. Consider the loss
\[L(\ta,\delta) = \sum_{i=1}^p(\ta_i-\delta_i(X))^2. \]
One can show (see Keener section 11) that the Bayes estimator is given by 
\[\delta_i(X) = \left(1-\frac{1}{1+\tau^2}\right)X_i = \frac{\tau^2}{1+\tau^2}X_i. \]
That is we shrink our data towards 0. The amount of shrinkage depends on $\tau^2$. If $\tau^2$ is small we shrink a lot, if it is large we shrink less. We will not prove that $\delta$ is the Bayes estimator but we will show that it has the lowest average risk of estimators of the form $\wh{\ta}_\al = \al X$ when $p=1$. The risk of such an estimator is
\[\E[(\ta-\al X)^2] = \E[\ta^2]-2\al\E[\ta X]+\al^2\E[X^2] = \tau^2-2\al \tau^2+\al^2(\tau^2+1), \]
which is minimized when $\al(\tau^2+1)=\tau^2$, that is $\al = \frac{\tau^2}{1+\tau^2}$. The shrinkage term is the ratio of the experiment noise and the measurement noise. 

What if we don't knw $\tau$? We can still estimate it. This is the idea behind empirical Bayes. We can estimate the variance of $X_i$ since 
\[X_i \iid N(0,1+\tau^2),  \]
Thus $\frac{1}{p}\sum_{j=1}^p X_j^2$ is unbiased for $1+\tau^2$. This gives us the new estimator
\[\delta'_i(X) = \left(1-\frac{p}{\sum_{j=1}^n X_j^2}\right)X_i. \]
This is an example of emprical Bayes when the prior is estimated from the data. We will revisit this estimator in the frequentist setting later in the course.
\subsection{Why Bayes estimators?}
\begin{enumerate}
    \item Every admissible estimator is a Bayes estimator or a limit of Bayes estimators. That is for a sequence of priors $\Lambda_n$, $\delta_{\Lambda_n}(x) \to \delta(x)$ a.e. $\Pa_\ta$.
    \item Can incorporate prior experience and beliefs.
    \item Quantification of uncertainty are sometimes easier to interpret.
    \item Encode complex data structures.
\end{enumerate}
How do we choose priors?
\begin{enumerate}
    \item Subjective: previous knowlegde.
    \item Objective: Select an ``uniformative prior'' such as a Jeffery's prior or a uniform prior.
    \item Empirical: Estimate prior parameters from data.
    \item Computational: Use conjugate priors.
\end{enumerate}
As $n \to \infty$, the posterior is independent of the prior - Bernstein-von-Mises.
\section{Minimax estimators}
Given $X \sim \Pa_\ta$, $\ta$ fixed and unknown. Our goal is to find an estimator $\delta$ that minimizes
\[\sup_{\ta \in \Om} R(\ta,\delta). \]
Such a $\delta$ is called a minimax estimator. How do we find such estimators? Note that 
\[\sup_\ta R(\ta,\delta) \ge \int_\Om R(\ta,\delta)d\Lambda, \]
for all probability distributions $\Lambda$. Thus our goal is to find the ``worst prior.''
\begin{defn}
    A prior $\Lambda$ is called \emph{least favourable} if $r_\Lambda \ge r_{\Lambda'}$ for any other prior $\Lambda'$.
\end{defn}
Recall that $r_{\Lambda} = r(\Lambda, \delta_\Lambda)$ where $\delta_\Lambda$ is the Bayes estimator for $\Lambda$.
\begin{thrm}
    \emph{[TPE 5.1.4]} Suppose $\delta_{\Lambda}$ is a Bayes estimator with $r_\Lambda = \sup_\ta R(\ta, \delta_\Lambda)$, then 
    \begin{enumerate}
        \item $\delta_\Lambda$ is minimax.
        \item $\Lambda$ is least favourable.
        \item If $\delta_\Lambda$ is a unique Bayes estimator (a.e $\Pa_\ta$ for all $\ta \in \Om$), then $\delta_\Lambda$ is the unique minimax estimator.
    \end{enumerate}
\end{thrm}
\begin{proof}
    For (a) let $\delta$ be another estimator, then 
    \begin{align*}
        \sup_\ta R(\ta,\delta) & \ge \int_\Om R(\ta,\delta)d\Lambda \\
        &\ge \int_\Om R(\ta, \delta_\Lambda)d\Lambda \\
        &= r_\Lambda\\
        &= \sup_\ta R(\ta,\delta_\Lambda)
    \end{align*}
    Thus $\delta_\Lambda$ is minimax. For (b), let $\Lambda'$ be another prior, then
    \begin{align*}
        r_{\Lambda'} & = \int_\Om R(\ta, \delta_{\Lambda'})d\Lambda'\\
        &\le \int_\Om R(\ta, \delta_{\Lambda})d\Lambda'\\
        &\le \sup_{\ta \in \Om} R(\ta, \delta_{\Lambda})\\
        &= \int_\Om R(\ta, \delta_{\Lambda})d\Lambda\\
        &= r_\Lambda.
    \end{align*}
    Thus $\Lambda$ is least favourable. For (c), suppose that $\delta$ is a minimax estimator, then 
    \begin{align*}
        r(\Lambda, \delta) &=\int_{\Om} R(\ta, \delta)d\Lambda \\
        &\le \sup_{\ta \in \La} R(\ta,\delta)\\
        &= \sup_{\ta \in \La}R(\ta,\delta_{\Lambda})\\
        &=r_{\Lambda}.
    \end{align*}
    Thus $\delta$ is the Bayes estimator of the prior $\Lambda$. It follows that if $\delta_{\Lambda}$ is the unique Bayes estimator for $\Lambda$, then $\delta_{\Lambda}$ is the unique minimax estimator.
\end{proof}
\begin{ex}
    Suppose $X \sim \text{Bin}(n,\ta)$ and we want to find the minimax estimator of $\ta$ under squared error loss. Our goal is to find a Bayes estimator that has constant risk. This would automatically give
    \[r_{\Lambda} = \sup_{\ta \in \Om} R(\ta,\delta_{\Lambda}). \]
    We previously say that if $\Ta$ has a Beta$(a,b)$ prior, then the Bayes estimator has the form 
    \[\delta_\Lambda(x) = \frac{x+a}{a+b+n}.\]
    We wish to find $a,b$ such that this estimator has constant risk. Note that
    \begin{align*}
        R(\ta,\delta_\Lambda) &= \E_\ta \left[\left(\frac{x+a}{n+a+b}-\ta\right)^2\right]\\
        &=\frac{1}{(n+a+b)^2}\E_\ta\left[\left(X+a-\ta(n+a+b)\right)^2\right]\\
        &=\frac{1}{(n+a+b)^2}\E_\ta\left[\left(X-n\ta +a(1-\ta)-b\ta\right)^2\right]\\
        &=\frac{1}{(n+a+b)^2}\left[\V_\ta(X-n\ta) +(a(1-\ta)-b\ta)^2\right]\\
        &=\frac{1}{(n+a+b)^2}\left[n\ta(1-\ta) +(a(1-\ta)-b\ta)^2\right].
    \end{align*} 
    Thus we wish to find $a,b$ such that $n\ta(1-\ta)+(a(1-\ta)-b\ta)^2$ is constant in $\ta$. The solution is $a=b=\frac{\sqrt{n}}{2}$. Thus
    \[\delta_\Lambda(x) = \frac{x+\frac{\sqrt{n}}{2}}{n+\sqrt{n}}, \]
    is a minimax estimator for $\ta$.
\end{ex}
\end{document}