\documentclass{article}
\usepackage{ae,aecompl}
\usepackage{todonotes}
\usepackage{chngcntr}
\usepackage{tikz-cd}
\usepackage{graphicx}
\graphicspath{ {./images/}}
\usepackage[all,cmtip]{xy}
\usepackage{amsmath, amscd}
\usepackage{amsthm}
\usepackage{amssymb}
\usepackage{amsfonts}
\usepackage{bm}
\usepackage{qsymbols}
\usepackage{latexsym}
\usepackage{mathrsfs}
\usepackage{mathtools}
\usepackage{cite}
\usepackage{color}
\usepackage{url}
\usepackage{enumerate}
\usepackage{verbatim}
\usepackage[draft=false, colorlinks=true]{hyperref}
\usepackage{pdfpages}
\usepackage[margin=1.2in]{geometry}
\usepackage{IEEEtrantools}
\usepackage{multirow}
\usepackage{fancyhdr}


\usepackage[nameinlink]{cleveref}


\DeclareMathOperator*{\ac}{accept}
\DeclareMathOperator*{\amax}{argmax}
\DeclareMathOperator*{\amin}{argmin}
\DeclareMathOperator*{\Aut}{Aut}
\newcommand {\al}{{\alpha}}
\newcommand {\abs}[1]{{\left\lvert#1\right\rvert}}
\newcommand {\A}{{\mathcal{A}}}
\newcommand {\AM}{{\mathrm{AM}}}
\newcommand {\AMp}{{\AM_{p}^{X}\!(\Ri_\w)}}
\newcommand {\B}{{\mathcal{B}}}
\DeclareMathOperator*{\Be}{Bern}
\newcommand {\Br}{{\dot{B}}}
\newcommand {\Ba}{{\mathfrak{B}}}
\newcommand {\C}{{\mathbb C}}
\newcommand {\ce}{\mathrm{c}}
\newcommand {\Ce}{\mathrm{C}}
\newcommand {\Cc}{\mathrm{C_{c}}}
\newcommand {\Ccinf}{\mathrm{C_{c}^{\infty}}}
\DeclareMathOperator{\DEV}{DEV}
\DeclareMathOperator{\diag}{diag}
\newcommand {\Di}{{\mathbb D}}
\newcommand {\dom}{\mathrm{dom}}
\newcommand {\ud}{\mathrm{d}}
\newcommand {\ue}{\mathrm{e}}
\newcommand {\eps}{\varepsilon}
\newcommand {\veps}{\varepsilon}
\newcommand {\vrho}{{\varrho}}
\newcommand {\E}{{\mathbb{E}}}
\newcommand {\Ec}{{\mathcal{E}}}
\newcommand {\Ell}{L}
\newcommand {\Ellp}{{L_{p}[0,1]}}
\newcommand {\Ellpprime}{{L_{p'}([0,1])}}
\newcommand {\Ellq}{{L_{q}([0,1])}}
\newcommand {\Ellqprime}{{L_{q'}([0,1])}}
\newcommand {\Ellr}{L^{r}}
\newcommand {\Ellone}{{L_{1}([0,1])}}
\newcommand {\Elltwo}{{L_{2}([0,1])}}
\newcommand {\Ellinfty}{L^{\infty}}
\newcommand {\Ellinftyc}{L_{\mathrm{c}}^{\infty}}
\newcommand {\exb}[1]{\exp\left\{#1\right\}}
\DeclareMathOperator*{\Exp}{Exp}
\DeclareMathOperator*{\Ext}{Ext}
\newcommand {\F}{{\mathcal{F}}}
\newcommand{\Fe}{{\mathbb{F}}}
\newcommand
{\G}{{\mathcal{G}}}
\newcommand {\HF}{\mathcal{H}_{\text{FIO}}^{1}(\Rd)}
\newcommand {\Hr}{H}
\newcommand {\HT}{\mathcal{H}}
\newcommand {\ui}{\mathrm{i}}
\newcommand {\I}{{I}}
\newcommand {\J}{{\mathcal{J}}}
\newcommand {\id}{{\mathrm{id}}}
\newcommand {\iid}{\stackrel{\mathclap{\normalfont\mbox{iid}}}{\sim}}
\newcommand {\im}{{\text{im }}}
\newcommand {\ind}{{\perp\!\!\!\perp}}
\newcommand{\indep}{\stackrel{\text{Indep}}{\sim}}
\DeclareMathOperator*{\Int}{int}
\newcommand {\intx}{{\overline{\int_{X}}}}
\newcommand {\inte}{{\overline{\int_{\E}}}}
\newcommand {\la}{\lambda}
\newcommand {\rb}{\rangle}
\newcommand {\lb}{{\langle}}
\newcommand {\La}{\Lambda}
\newcommand {\calL}{{\mathcal{L}}}
\newcommand {\lp}{{\mathcal{L}}^{p}}
\newcommand {\lpo}{{\overline{\mathcal{L}}^{p}\!}}
\newcommand {\Lpo}{{\overline{\Ell}^{p}\!}}
\newcommand {\M}{{\mathbf{M}}}
\newcommand {\Ma}{{\mathcal{M}}}
\newcommand {\N}{{{\mathbb N}}}
\newcommand {\Na}{{{\mathcal{N}}}}
\newcommand {\norm}[1]{\left\|#1\right\|}
\newcommand {\normm}[1]{{\left\vert\kern-0.25ex\left\vert\kern-0.25ex\left\vert #1 
    \right\vert\kern-0.25ex\right\vert\kern-0.25ex\right\vert}}
\newcommand {\Om}{{{\Omega}}}
\newcommand {\one}{{{\bf 1}}}
\newcommand {\pic}{\text{Pic }}
\newcommand {\ph}{{\varphi}}
\newcommand {\Pa}{{\mathbb{P}}}
\newcommand {\Po}{{\mathcal{P}}}
\newcommand {\Q}{{\mathbb{Q}}}
\newcommand {\R}{{\mathbb R}}
\newcommand {\Rd}{{\mathbb{R}^{d}}}
\DeclareMathOperator{\rej}{reject }
\newcommand {\Rn}{{\mathbb{R}^{n}}}
\newcommand {\cR}{{\mathcal{R}}}
\newcommand {\Rad}{{\mathrm{Rad}}}
\newcommand {\ran}{{\mathrm{ran}}}
\newcommand {\Ri}{{\mathrm{R}}}
\newcommand {\supp}{{\mathrm{supp}}}
\newcommand {\Se}{\mathrm{S}}
\newcommand {\Sp}{S^{*}(\Rn)}
\newcommand {\St}{{\mathrm{St}}}
\newcommand {\Sw}{\mathcal{S}}
\newcommand {\T}{{\mathcal{T}}}
\newcommand {\ta}{{\theta}}
\newcommand {\Ta}{{\Theta}}
\DeclareMathOperator {\V}{Var}
\newcommand {\w}{{\omega}}
\newcommand {\W}{{\mathrm{W}}}
\newcommand {\Wnp}{\text{$\mathrm{W}$\textsuperscript{$n,\!p$}}}
\newcommand {\Wnpeq}{\text{$\mathrm{W}$\textsuperscript{$n\!,\!p$}}}
\newcommand {\Wonep}{\text{$\mathrm{W}$\textsuperscript{$1,\!p$}}}
\newcommand {\Wonepeq}{\text{$\mathrm{W}$\textsuperscript{$1\!,\!p$}}}
\newcommand {\X}{{\mathcal{X}}}
\newcommand {\Z}{{{\mathbb Z}}}
\newcommand {\Za}{{\mathcal{Z}}}
\newcommand {\Zd}{{\Z[\sqrt{d}]}}
\newcommand {\vanish}[1]{\relax}

\newcommand {\wh}{\widehat}
\newcommand {\wt}{\widetilde}
\newcommand {\red}{\color{red}}

% Distributions
\newcommand{\normal}{\mathsf{N}}
\newcommand{\poi}{\mathsf{Poisson}}
\newcommand{\bern}{\mathsf{Bernoulli}}
\newcommand{\bin}{\mathsf{Binomal}}
\newcommand{\multi}{\mathsf{Multinomial}}




% put your command and environment definitions here




% some theorem environments
% remove "[theorem]" if you do not want them to use the same number sequence


  \newtheorem{thrm}{Theorem}
  \newtheorem{lemma}{Lemma}
  \newtheorem{prop}{Proposition}
  \newtheorem{cor}{Corollary}

  \newtheorem{conj}{Conjecture}
  \renewcommand{\theconj}{\Alph{conj}}  % numbered A, B, C etc

  \theoremstyle{definition}
  \newtheorem{defn}{Definition}
  \newtheorem{ex}{Example}
  \newtheorem{exs}{Examples}
  \newtheorem{question}{Question}
  \newtheorem{remark}{Remark}
  \newtheorem{notn}{Notation}
  \newtheorem{exer}{Exercise}

\newcommand{\Y}{\mathcal{Y}}
\newcommand{\Cyl}{\mathcal{C}}
\newcommand{\U}{\mathcal{U}}


\title{STATS310A - Lecture 10}
\author{Persi Diaconis\\ Scribed by Michael Howes}
\date{10/21/21}

\pagestyle{fancy}
\fancyhf{}
\rhead{STATS310A - Lecture 10}
\lhead{10/21/21}
\rfoot{Page \thepage}

\begin{document}
\maketitle
\tableofcontents
\section{Product $\sigma$-algebras}
Let $(X,\X)$ and $(Y,\Y)$ be two measurable spaces. Let $X \times Y$ be the product \emph{set} $X \times Y = \{(x,y): x \in X, y \in Y\}$. Define the \emph{projections} $\pi_X : X\times Y \to X$ and $\pi_Y : X \times Y \to Y$ by
\[\pi_X(x,y) = x ~~\text{and}~~ \pi_Y(x,y) = y. \]
\begin{defn}
    The \emph{product $\sigma$-algebra} is the small $\sigma$-algebra on $X \times Y$ making $\pi_X$ and $\pi_Y$ measurable. We denote the product $\sigma$-algebra by $\X \times \Y$.
\end{defn}
\begin{defn}
    The \emph{cyclinder sets} are sets of the form $\pi_X^{-1}(A)$ for $A \in \X$ or $\pi_Y^{-1}(B)$ for $B \in \Y$. We denote the class of cyclinder sets by $\Cyl$.
\end{defn}
\begin{defn}
    Let $\Po = \{A \times B : A \in \X, B \in \Y\}$ be the class of \emph{measurable rectangles}.
\end{defn}
Note that $\Po$ is a $\pi$-system and indeed a semi-ring. Define $\U$ to the set of finite disjoint unions of measurable rectangles. The collection $U$ is a field.

\begin{prop}
    With the notation as above
    \[\X \times \Y = \sigma(\pi_X,\pi_Y) = \sigma(\Cyl) = \sigma(\Po) = \sigma(U).  \]
\end{prop}

\begin{defn}
    If $A \subset X \times Y$ and $x \in X$, define
    \[A_x = \{y:(x,y) \in A\} \subseteq Y. \]
    For $y \in Y$, define
    \[A_y = \{x : (x,y) \in A\} \subseteq X.\]
    The sets $A_x$ and $A_y$ are called \emph{sections} of $A$. 
\end{defn}
\begin{defn}
    For a function $F:X \times Y \to W$, define $f_x : Y \to W$ and $f_y : X \to W$ by 
    \[f_x(y) = f(x,y) ~~\text{and}~~f_y(x) = f(x,y). \]
    The maps $f_x$ and $f_y$ are again called \emph{sections} of $f$.
\end{defn}
\begin{prop}
    Sections commute with set operations. That is
    \begin{itemize}
        \item $(A^c)_x = A_x^c$,
        \item $\left(\bigcap_{i \in I} A^i\right)_x = \bigcap_{i \in I} A^i_x$,
        \item $\left(\bigcup_{i \in I} A^i\right)_x = \bigcup_{i \in I} A^i_x$,
    \end{itemize}
    where $I$ is any index set.
\end{prop}
\begin{prop}
    If $A \in \X \times Y$, then $A_x \in \Y$ for all $x \in X$. If $f:X \times Y \to (W,\F)$ is measurable, then $f_x:Y \to (W,\F)$ and $f_y : X \to (W,\F)$ are also measurable.
\end{prop}
\begin{proof}
    Consider the collection 
    \[G = \{A \in \X \times \Y : A_x \in \Y\}. \]
    Note that $G$ contains the measurable rectangles since 
    \[(R\times S)_x = \begin{cases}
        \emptyset& \text{if } x \notin R,\\
        S & \text{if } x \in R.
    \end{cases}\]
    Thus in either case $(R \times S)_x \in \Y$. Since sections commute with set operations, $G$ is a $\sigma$-algebra. Thus $\sigma(\Po) = \X \times \Y \subseteq G$, as required.

    Let $A$ be a measurable subset of $W$. Then 
    \begin{align*}
        f_x^{-1}(A) &=\{y:f_x(y) \in A\}\\
        &= \{y:f(x,y) \in A\}\\
        &=\{y: (x,y) \in f^{-1}(A)\}\\
        &=(f^{-1}(A))_x.
    \end{align*}
    Since $f^{-1}(A) \in \X \times \Y$, we can conclude that $f_x^{-1}(A) = (f^{-1}(A))_x \in \Y$.
\end{proof}
\section{Measures on product spaces}
\begin{defn}
    Let $(X,\X)$ and $(Y,\Y)$ be measurable spaces. A \emph{Markov Kernel} is a function $K : X \times \Y \to [0,1]$ such that 
    \begin{enumerate}
        \item For all $x \in X$, $K(x,\cdot)$ is a probability measure on $(Y,\Y)$.
        \item For all $B \in \Y$, $K(\cdot, B)$ is measurable.
    \end{enumerate}
    We will write $K(x,dy)$ to mean that $K$ is a Markov kernel $K:X\times \Y \to [0,1]$.
\end{defn}
\begin{ex}
    Say $\nu$ is a probability measure on $(Y,\Y)$, then $K(x,B) = \nu(B)$ is a Markov Kernel.
\end{ex}
\begin{ex}
    If $\Theta$ is any set and $\F$ a $\sigma$-algebra on $\Theta$, then a family of probabilities $\{\Pa_\theta(\cdot):\theta \in \Theta\}$  on $(X,\X)$ is a Markov kernel
    \[K(\theta,B) = \Pa_\theta(B). \]
\end{ex}
\begin{ex}
    If $X=Y$, then $k(x,dy)$ defines a \emph{Markov chain} on $X$.
\end{ex}
\begin{defn}
    Let $(X,\X)$ and $(Y,\Y)$ be measurable spaces and $K(x,dy)$ a kernel and $\mu$ a probability on $X$. The product measure $\mu \times K$  is a set function on $\X \times \Y$ defined by 
    \[\mu\times K (A) = \int_X K(x,A_x)\mu(dx). \]
\end{defn}
\begin{prop}
    The mapping $x \mapsto K(x,A_x)$ is measurable and integrable. Furthermore $\mu \times K$ is a probability on $X \times Y$.
\end{prop}
\begin{proof}
    Define 
    \[G = \{A \in \X \times \Y : x \mapsto K(x,A_x) \text{ is measurable}\}. \]
    Note that $G$ contains the measurable rectangles. This is because 
    \begin{align*}
        K(x,(S\times R)_x) &=\begin{cases}
            0 & \text{if } x \notin S,\\
            K(x,R) & \text{if } x \in S.   
        \end{cases}\\
        &= \delta_S(x)K(x,R).
    \end{align*}
    Thus $x \mapsto K(x,(S\times R)_x)$ is the product of two measurable functions and hence measurable. Thus $G$ continas the $\pi$-system $\Po$. We will now show that $G$ is a $\la$-system. Note that $X \times Y \in G$, since $X \times Y \in \Po$. Furthermore if $A \in G$, then 
    \[K(x,(A^c)_x) = K(x,A_x^c) = 1-K(x,A_x), \]
    and so $A^c \in G$. Finally if $(A^i)_{i=1}^\infty$ are disjoint, then $(A^i_x)_{i=1}^\infty$ are disjoint and hence
    \[ K\left(x,\left(\bigcup_{i=1}^\infty A^i\right)_x\right) = K\left(x,\bigcup_{i=1}^\infty A_x^i\right) =\sum_{i=1}^\infty K(x,A_x^i),\]
    and thus $\bigcup_i A^i \in G$ since the limits of measurable functions are measurable. Thus $G$ is a $\lambda$-system and it must contain $\sigma(\Po) = \X \times \Y$ by the $\pi$-$\lambda$ theorem.

    To see that $\mu \times K$ is a probability measure one can use the monotone convergence theorem.
\end{proof}
\begin{ex}
    If $K(x,B) = \nu(B)$ then we write $\mu \times K$ as $\mu \times \nu$ and call $\mu \times \nu$ the product measure.
\end{ex}
\begin{ex}
    [Decision theory/Bayesian statistics] Given probability distributions $P=\{\Pa_\theta(\cdot)\}_{\theta \in \Theta}$ on $(X,\X)$ and a probability $\pi$ on $\Theta$, $\pi \times P$ defines a probability on $\Theta \times X$. Define 
    \[m(B) = \int_\Theta \Pa_\theta(B)\pi(d\theta), \]
    which is a probability distribution on $(X,\X)$ called the \emph{marginal distribution}. A \emph{posterior} is a kernel $K(x,d\theta)$ on $X \times \F_\theta$ such that
    \[\int_A P_\theta(B)\pi(d\theta) = \int_B K(x,A)\pi(dx), \]
    for all $A \in \F_\theta$ and $B \in \X$. Unfortunately posteriors don't always exist. \grumpy We need topological conditions on $X$ to be sure that posteriors exist (eg it suffices for $X$ to be a complete seperable metric space). When things work out the objects of study are called regular conditional probabilities.
\end{ex}
\section{Fubinni's Theorem}
\begin{thrm}
    Let $(X,\X)$ and $(Y,\Y)$ be measurable spaces. Let $\mu(dx)$ be a measure and $K(x,dy)$ be a kernel. The if $f:X\times Y \to [0,\infty]$ is measurable, then 
    \[x \mapsto \int_Y f(x,y)K(x,dy),\]
    is measurable on $(X,\X)$ and 
    \[\int_{X\times Y} f(x,y)(\mu \times K)(dx,dy) = \int_X\left(\int_Y f(x,y)K(x,dy)\right)\mu(dx). \]
\end{thrm}
\begin{proof}
    We will use a (1), (2), (3) argument. Let $G$ be the set of all measurable $f:X \times Y \to \R^+$ such that the above two results hold. Suppose that $A \in \X\times Y$ and $f=\delta_A$. Then note that $\delta_A(x,y) = \delta_{A_x}(y)$ and so
    \[\int_Y \delta_A(x,y)K(x,dy) = \int_Y \delta_{A_x}(y)K(x,dy) = K(x,A_x), \]
    which is measurable. And furthermore 
    \begin{align*}
        \int_{X\times Y} \delta_{A}(x,y)(\mu \times K)(dx,dy)&=(\mu\times K)(A)\\
        & = \int_X K(x,A_x)\mu(dx)\\
        &=\int_X\left(\int_Y \delta_A(x,y)K(x,dy)\right)\mu(dx).
    \end{align*}
    Thus $\delta_A \in G$. One can check that $G$ is closed under linear combinations and monotone limits. Thus $G$ contains all non-negative measurable functions. 
\end{proof}
\begin{remark}
    \begin{enumerate}
        \item We assumed $K(\cdot,B)$ and $\mu(\cdot)$ where probability measures. Everything works under the more general assumption that $K(\cdot,B)$ and $\mu(\cdot)$ are $\sigma$-finite.
        \item The textbook carefully works through the case when $K(x,dy) = \nu(dy)$.
        \item When applying Fubinni's theorem look out for functions that are both positive and negative. Everything works if $\int \abs{f}(\mu\times K)(dx,dy) < \infty$.
        \item These results do not hold for finitely additive measures or non $\sigma$-finite measures.
        \item Measures on infinite products require care. You again need topology to deal with something like
         \[\mu(x_1), K(x_1dx_2), L((x_1,x_2), dx_3), \ldots \]
    \end{enumerate}
\end{remark}
\end{document}