\documentclass{article}
\usepackage{ae,aecompl}
\usepackage{todonotes}
\usepackage{chngcntr}
\usepackage{tikz-cd}
\usepackage{graphicx}
\graphicspath{ {./images/}}
\usepackage[all,cmtip]{xy}
\usepackage{amsmath, amscd}
\usepackage{amsthm}
\usepackage{amssymb}
\usepackage{amsfonts}
\usepackage{bm}
\usepackage{qsymbols}
\usepackage{latexsym}
\usepackage{mathrsfs}
\usepackage{mathtools}
\usepackage{cite}
\usepackage{color}
\usepackage{url}
\usepackage{enumerate}
\usepackage{verbatim}
\usepackage[draft=false, colorlinks=true]{hyperref}
\usepackage{pdfpages}
\usepackage[margin=1.2in]{geometry}
\usepackage{IEEEtrantools}
\usepackage{multirow}
\usepackage{fancyhdr}


\usepackage[nameinlink]{cleveref}


\DeclareMathOperator*{\ac}{accept}
\DeclareMathOperator*{\amax}{argmax}
\DeclareMathOperator*{\amin}{argmin}
\DeclareMathOperator*{\Aut}{Aut}
\newcommand {\al}{{\alpha}}
\newcommand {\abs}[1]{{\left\lvert#1\right\rvert}}
\newcommand {\A}{{\mathcal{A}}}
\newcommand {\AM}{{\mathrm{AM}}}
\newcommand {\AMp}{{\AM_{p}^{X}\!(\Ri_\w)}}
\newcommand {\B}{{\mathcal{B}}}
\DeclareMathOperator*{\Be}{Bern}
\newcommand {\Br}{{\dot{B}}}
\newcommand {\Ba}{{\mathfrak{B}}}
\newcommand {\C}{{\mathbb C}}
\newcommand {\ce}{\mathrm{c}}
\newcommand {\Ce}{\mathrm{C}}
\newcommand {\Cc}{\mathrm{C_{c}}}
\newcommand {\Ccinf}{\mathrm{C_{c}^{\infty}}}
\DeclareMathOperator{\DEV}{DEV}
\DeclareMathOperator{\diag}{diag}
\newcommand {\Di}{{\mathbb D}}
\newcommand {\dom}{\mathrm{dom}}
\newcommand {\ud}{\mathrm{d}}
\newcommand {\ue}{\mathrm{e}}
\newcommand {\eps}{\varepsilon}
\newcommand {\veps}{\varepsilon}
\newcommand {\vrho}{{\varrho}}
\newcommand {\E}{{\mathbb{E}}}
\newcommand {\Ec}{{\mathcal{E}}}
\newcommand {\Ell}{L}
\newcommand {\Ellp}{{L_{p}[0,1]}}
\newcommand {\Ellpprime}{{L_{p'}([0,1])}}
\newcommand {\Ellq}{{L_{q}([0,1])}}
\newcommand {\Ellqprime}{{L_{q'}([0,1])}}
\newcommand {\Ellr}{L^{r}}
\newcommand {\Ellone}{{L_{1}([0,1])}}
\newcommand {\Elltwo}{{L_{2}([0,1])}}
\newcommand {\Ellinfty}{L^{\infty}}
\newcommand {\Ellinftyc}{L_{\mathrm{c}}^{\infty}}
\newcommand {\exb}[1]{\exp\left\{#1\right\}}
\DeclareMathOperator*{\Exp}{Exp}
\DeclareMathOperator*{\Ext}{Ext}
\newcommand {\F}{{\mathcal{F}}}
\newcommand{\Fe}{{\mathbb{F}}}
\newcommand
{\G}{{\mathcal{G}}}
\newcommand {\HF}{\mathcal{H}_{\text{FIO}}^{1}(\Rd)}
\newcommand {\Hr}{H}
\newcommand {\HT}{\mathcal{H}}
\newcommand {\ui}{\mathrm{i}}
\newcommand {\I}{{I}}
\newcommand {\J}{{\mathcal{J}}}
\newcommand {\id}{{\mathrm{id}}}
\newcommand {\iid}{\stackrel{\mathclap{\normalfont\mbox{iid}}}{\sim}}
\newcommand {\im}{{\text{im }}}
\newcommand {\ind}{{\perp\!\!\!\perp}}
\newcommand{\indep}{\stackrel{\text{Indep}}{\sim}}
\DeclareMathOperator*{\Int}{int}
\newcommand {\intx}{{\overline{\int_{X}}}}
\newcommand {\inte}{{\overline{\int_{\E}}}}
\newcommand {\la}{\lambda}
\newcommand {\rb}{\rangle}
\newcommand {\lb}{{\langle}}
\newcommand {\La}{\Lambda}
\newcommand {\calL}{{\mathcal{L}}}
\newcommand {\lp}{{\mathcal{L}}^{p}}
\newcommand {\lpo}{{\overline{\mathcal{L}}^{p}\!}}
\newcommand {\Lpo}{{\overline{\Ell}^{p}\!}}
\newcommand {\M}{{\mathbf{M}}}
\newcommand {\Ma}{{\mathcal{M}}}
\newcommand {\N}{{{\mathbb N}}}
\newcommand {\Na}{{{\mathcal{N}}}}
\newcommand {\norm}[1]{\left\|#1\right\|}
\newcommand {\normm}[1]{{\left\vert\kern-0.25ex\left\vert\kern-0.25ex\left\vert #1 
    \right\vert\kern-0.25ex\right\vert\kern-0.25ex\right\vert}}
\newcommand {\Om}{{{\Omega}}}
\newcommand {\one}{{{\bf 1}}}
\newcommand {\pic}{\text{Pic }}
\newcommand {\ph}{{\varphi}}
\newcommand {\Pa}{{\mathbb{P}}}
\newcommand {\Po}{{\mathcal{P}}}
\newcommand {\Q}{{\mathbb{Q}}}
\newcommand {\R}{{\mathbb R}}
\newcommand {\Rd}{{\mathbb{R}^{d}}}
\DeclareMathOperator{\rej}{reject }
\newcommand {\Rn}{{\mathbb{R}^{n}}}
\newcommand {\cR}{{\mathcal{R}}}
\newcommand {\Rad}{{\mathrm{Rad}}}
\newcommand {\ran}{{\mathrm{ran}}}
\newcommand {\Ri}{{\mathrm{R}}}
\newcommand {\supp}{{\mathrm{supp}}}
\newcommand {\Se}{\mathrm{S}}
\newcommand {\Sp}{S^{*}(\Rn)}
\newcommand {\St}{{\mathrm{St}}}
\newcommand {\Sw}{\mathcal{S}}
\newcommand {\T}{{\mathcal{T}}}
\newcommand {\ta}{{\theta}}
\newcommand {\Ta}{{\Theta}}
\DeclareMathOperator {\V}{Var}
\newcommand {\w}{{\omega}}
\newcommand {\W}{{\mathrm{W}}}
\newcommand {\Wnp}{\text{$\mathrm{W}$\textsuperscript{$n,\!p$}}}
\newcommand {\Wnpeq}{\text{$\mathrm{W}$\textsuperscript{$n\!,\!p$}}}
\newcommand {\Wonep}{\text{$\mathrm{W}$\textsuperscript{$1,\!p$}}}
\newcommand {\Wonepeq}{\text{$\mathrm{W}$\textsuperscript{$1\!,\!p$}}}
\newcommand {\X}{{\mathcal{X}}}
\newcommand {\Z}{{{\mathbb Z}}}
\newcommand {\Za}{{\mathcal{Z}}}
\newcommand {\Zd}{{\Z[\sqrt{d}]}}
\newcommand {\vanish}[1]{\relax}

\newcommand {\wh}{\widehat}
\newcommand {\wt}{\widetilde}
\newcommand {\red}{\color{red}}

% Distributions
\newcommand{\normal}{\mathsf{N}}
\newcommand{\poi}{\mathsf{Poisson}}
\newcommand{\bern}{\mathsf{Bernoulli}}
\newcommand{\bin}{\mathsf{Binomal}}
\newcommand{\multi}{\mathsf{Multinomial}}




% put your command and environment definitions here




% some theorem environments
% remove "[theorem]" if you do not want them to use the same number sequence


  \newtheorem{thrm}{Theorem}
  \newtheorem{lemma}{Lemma}
  \newtheorem{prop}{Proposition}
  \newtheorem{cor}{Corollary}

  \newtheorem{conj}{Conjecture}
  \renewcommand{\theconj}{\Alph{conj}}  % numbered A, B, C etc

  \theoremstyle{definition}
  \newtheorem{defn}{Definition}
  \newtheorem{ex}{Example}
  \newtheorem{exs}{Examples}
  \newtheorem{question}{Question}
  \newtheorem{remark}{Remark}
  \newtheorem{notn}{Notation}
  \newtheorem{exer}{Exercise}


\title{STATS 305A - Lecture 2}
\author{John Duchi\\ Notes by Michael Howes}

\begin{document}
\maketitle
\section{Outline}
Today we will do two things. A matrix and linear algebra overview. Covering
\begin{enumerate}
    \item Independence,
    \item Rank/ invertibility/ solving $Ax=b$,
    \item Decompositions.
\end{enumerate}
We will also discuss some basic of optimisation. Detailed linear algebra notes will be put on the course webpage.

\section{Linear algebra review}
\subsection{Vectors}
Vector are $x \in \R^n$, we will write
\[x = \begin{bmatrix}
    x_1\\
    x_2\\
    \vdots\\
    x_n
\end{bmatrix}, \quad x_i \in \R.\]
We will not use row vectors. All vectors are column vectors. The \emph{norm} of a vector is its size. We will must commonly use the Euclidean or 2-norm. That is we will define
\[\norm{x} = \sqrt{\sum_{i=1}^n x_i^2}=\sqrt{x^Tx}. \]
We will occassionally use $p$-norms for $p \in [1,\infty]$ given by
\[\norm{x}_p = \sqrt[p]{\sum_{i=1}^n \abs{x_i}^p}. \]
When $p=1$, we have $\norm{x}_1 = \sum_{i=1}^n \abs{x_i}$. When $p=2$, $\norm{x}_2$ is the Euclidean norm and when $p=\infty$, $\norm{x}_\infty = \max\{\abs{x_i}: i =1,\ldots, n\}$. These are the values of $p$ we will most commonly use.

We will say that $k$-vectors $x_1,\ldots, x_k$ are \emph{(linearly) independent} if for all $\al \in \R^k$,
\[\sum_{i=1}^k \al_ix_i = 0 \iff \al_i = 0 \text{ for } i=1,2,\ldots,k. \]
This is equivalent to requiring that for all $\al, \eta \in \R^k$,
\[\sum_{i=1}^k \al_ix_i = \sum_{i=1}^k \beta_ix_i \iff \al_i = \beta_i \text{ for } i = 1,2\ldots, k. \]
The \emph{linear span} of the collection of vectors $\{x_i\}_{i=1}^k$ is the set
\[\spn\{x_i\} = \left\{\sum_{i=1}^k\al_ix_i : \al \in \R^k\right\}. \]
Note that a collection of vectors $\{x_i\}_{i=1}^k$ are independent if and only if for $i=1,\ldots,k$, $x_i$ is not in $\spn\{x_j\}_{j\neq i}$.
\subsection{Matrices}
We will now look at solving and understanding the matrix equation $Ax=b$. This does not always have a solution. Suppose $A \in \R^{m\times n}$, then we have
\begin{IEEEeqnarray*}{rCl}
    A&=&[a_1,\ldots,a_n], \quad a_i \in \R^m,\\
    &=&\begin{bmatrix}
        \wt{a}_1^T\\
        \ldots\\
        \wt{a}_m^T
    \end{bmatrix},\quad \wt{a}_j \in \R^n.
\end{IEEEeqnarray*}
The vectors $a_i$ are the columns of $A$ and the transposed vectors $\wt{a}_j^T$ are the rows of $A$. The \emph{rank} of $A$ is 
\[\#\text{ of independent columns of A} = \# \text{ of independent rows of A}.\]
The rank of a matrix is a very unstable quality. We may have rank 0, then add a small amount of noise and suddenly have rank $n$. This makes it not very useful in statistics but we will use it occassionally.

Given $A\in\R^{m \times n}$, we say that $A$ has a \emph{left inverse} $B \in \R^{n \times m}$ such that 
\[BA = I_n = \begin{bmatrix}
    1&0&\ldots&0\\
    0&1&\ldots&0\\
    \vdots&\vdots&\ddots&\vdots\\
    0&0&\ldots&1
\end{bmatrix}. \]
The matrix $I_n$ is called the $n\times n$ identity. We say that $A$ has a right inverse $C \in \R^{n \times m}$ if $AC = I_m$. If $A$ has a left inverse $B$ and a right inverse $C$, then $B=C$ and we define $A^{-1}:=B=C$. In this case we say that $A$ is invertible with inverse $A^{-1}$. To see why $B=C$, note that
\[B = BI_m = B(AC) = (BA)C = I_nC = C. \]
\begin{defn}
    The \emph{range} or \emph{column space} of a matrix $A$ is equal to
    \[\spn\{a_i\}_{i=1}^n =\{Ax : x \in \R^n\} \subseteq \R^m. \]
    We have equality in the above line since $Ax = \sum_{i=1}^n a_ix_i$.
\end{defn}
\begin{defn}
    The \emph{null space} of $A$ is 
    \[\nll(A) = \{x \in \R^n : Ax = 0 \} \subseteq \R^n.\]
\end{defn}
The null space will be important in our linear models. If $Y = X\beta + \varepsilon$, then $\nll(X)$ contains all the directions in $\beta$ that we get no information about. Even if $\nll(X) = \{0\}$, there may be problems if our matrix is ``ill-conditioned''. This means that there are vectors that are ``close to'' $\nll(X)$. These are directions of $\beta$ in which it is hard (but not impossible) to get information.

\begin{thrm}
    \emph{[The rank-nullity theorem]} Let $A \in \R^{n \times m}$, then $\nll(A) = \range(A^T)^\perp$, where for $S \subseteq \R^n$, $S^\perp = \{y \in \R^n : y^Tx = 0, \text{ for all } x \in S\}$.
\end{thrm}

\begin{proof}
    \emph{(sketch)} If $x \in \nll(A)$, then $Ax = 0$. Thus for $y = A^Tw\in \range(A^T)$ we have 
    \[y^Tx= w^T(A^T)^Tx=w^T(Ax)=w^T0=0. \]
    Thus $y \in \range(A^T)^\perp$. If $x \in \range(A^T)^\perp$, then for all $w \in \R^n$,
    \[0 = (A^Tw)^Tx=w^T(Ax), \]
    which implies $Ax=0$. That is $x \in \nll(A)$.
\end{proof}
\subsection{Special matrices and matrix decompositions}
\begin{defn}
    A matrix $Q \in \R^{n \times n}$ is \emph{orthogonal} if $Q^TQ = QQ^T = I_n$. If $Q =[q_1,q_2,\ldots,q_n]$, then this is equivalent to 
    \[q_i^Tq_j = \begin{cases}
        1 & \text{if } i = j,
        0 ^ \text{else.} 
    \end{cases}\]
    That is the vectors $\{q_1,\ldots, q_n\}$ are an orthonormal set. \emph{[Aside: if $Q \in \R^{n \times n}$ and $Q^*Q = I$, then $Q$ is said to be \emph{orthonormal} or \emph{unitary}]}. 
    
    If $U \in \R^{n \times m}$ where $m \ge n$ (more rows than columns, tall matrix), then $U$ will also be called orthogonal if $UU^T = I_n$. Note that if $m > n$, then $U^TU$ cannot equal $I_m$. Again this is equivalent to the columns of $U$ being orthogonal.
\end{defn}
We now will examine a number of matrix factorisations. A lot of computational matrix calculations/statistics/ML involves reducing a matrix $A$ to a product of simpler matrices. In particular we can use these to solve the equation $Ax =b $. 

\begin{defn}
    A matrix $R \in \R^{n \times n}$ is \emph{upper (or right) triangular} if $R$ is of the form 
    \[R = \begin{bmatrix}
        r_{1,1} & r_{1,2} & \ldots & r_{1,n}\\
        0&r_{2,2} & \ldots & r_{2,n}\\
        \vdots&\vdots&\ddots&\vdots\\
        0&0&\ldots&r_{n,n}
    \end{bmatrix}. \]
    That is $R$ has all zeros below the main diagonal.
\end{defn}
If we are asked to solve $Rx=b$ and $R$ is upper triangular we can use the following ``back-subsitution'' algorithm. First write the equation as
\[\begin{bmatrix}
    r_{1,1} & r_{1,2} & \ldots&r_{1,n-1} & r_{1,n}\\
    0&r_{2,2} & \ldots &r_{2,n-1}& r_{2,n}\\
    \vdots&\vdots&\ddots&\vdots&\vdots\\
    0&0&\ldots&r_{n-1,n-1}&r_{n-1,n}\\
    0&0&\ldots&0&r_{n,n}
\end{bmatrix} \begin{bmatrix}
    x_1\\x_2\\ \vdots\\x_{n-1}\\x_n
\end{bmatrix}=\begin{bmatrix}
    b_1\\b_2\\ \vdots\\b_{n-1}\\b_n
\end{bmatrix}\]
Assume that $r_{i,i} \neq 0$ for all $i = 1,\ldots,n$. Thus we can see that we must have $r_{n,n}x_n = b_n$. Thus $x_n = \frac{b_n}{r_{n,n}}$. We can then move up to the next row where we have
\begin{IEEEeqnarray*}{rCl}
    r_{n-1,n_1}x_{n-1} + r_{n-1,n}x_n &=& b_{n-1}\\
    \therefore x_{n-1} &=& \frac{1}{r_{n-1,n-1}}\left(b_{n-1}-r_{n-1,n}x_n\right)\\
    &=& \frac{1}{r_{n-1,n-1}}\left(b_{n-1}-r_{n-1,n}\frac{b_n}{r_{n,n}}\right),
\end{IEEEeqnarray*}
since we previously showed $x_n = \frac{b_n}{r_{n,n}}$. Continuing in this way we can see that if we have solved for $x_n,x_{n-1},\ldots,x_{k-1}$ we can solve for $x_k$. This takes approximately $n^2$ opperations to perform which is pretty much optimal. 

What do we do if we have a matrix $A$ that isn't upper triangular? We can use the QR factorisation.

\begin{thrm}
    Let $A \in \R^{m \times n}$ be a matrix with $m \ge n$ ($A$ is a tall matrix), then there exist $Q \in \R^{m \times n}$ and $R \in \R^{n \times n}$ such that $R$ is upper triangular and invertible, $Q^TQ = I_n$ and $A = QR$. The matrices $Q$ and $R$ are called the \emph{QR factorisation of $A$}.
\end{thrm}
Note that if we have the QR factorisation of $A$, then we can easily solve $Ax = b$. Since this is because $Ax = b$ implies $Q^TAx=Q^Tb$. Since $A = QR$ we have $Q^TA = Q^TQR = R$ and thus the solution to $Rx=Q^Tb$ solves the original equation $Ax = b$. The QR factorisaation can be constructed iteratively via the Gram-Schmidt algorithm. 

Given $A = [a_1,a_2,\ldots,a_n] \in \R^{m \times n}$ and $k \le n$ we will produce 
\[Q_k=[q_1,\ldots,q_k] \in \R^{m\times k} \text{ and } R_k \in \R^{k \times k},  \]
such that 
\begin{itemize}
    \item $R_k$ is upper triangular, 
    \item $\spn\{q_i\}_{i=1}^k = \spn\{a_i\}_{i=1}^k$, 
    \item $Q^T_kQ_k = I_k$, and 
    \item $Q_kR_k = [a_1,a_2,\ldots, a_k]$. 
\end{itemize}    
When we reach $k=n$ we will be done. We will show how to do this for $k=1,2$ and generalise.

\underline{Start $k=1$}: Define $q_1 = \frac{a_1}{\norm{a_1}}$, $r_{11} = \norm{a_11}$, then $r_{11}q_1 = a_1$ and $q_1^Tq_1 = \frac{a_1^Ta_1}{\norm{a_1}^2}=1$.

\underline{Next step $k=2$}: We already have $q_1$ and we want $q_2$. We want to preserve the span and so $q_2$ should be a linear combination of $a_2$ and $q_1$. We also want $q_2$ to be orthogonal to $q_1$. Thus we subtract off the $q_1$ part of $a$ and define 
\[v = a_2 - a_2^Tq_1 q_1. \]
Then $v^Tq_1 = a_2^Tq_1 - a_2^Tq_1 q_1^Tq_1 = 0$. We want $q_2$ to have norm 1 and thus define $q_2 = \frac{v}{\norm{v}}$. Note that the equation $a_2 = \norm{v}q_2+a_2^Tq_1q_1$ shows that $r_{2,2} = \norm{v}$ and $r_{1,2} = a_2^Tq_1$. Thus we have constructed $R_2$ as well.

\underline{Inductive case}: Suppose that we have just finished the $k-1$ step. Generalising what we did before we can set
\[v = q_k - \sum_{i=1}^{k-1}a_k^Tq_iq_i. \]
Then $v^Tq_i = 0$ for $i=1,\ldots,k-1$ and we can define $q_k = \frac{v}{\norm{v}}$. Since again $a_k = \norm{v}q_k + \sum_{i=1}^{k-1}a_k^Tq_iq_i$, we have that
\[r_{i,k} = \begin{cases}
    a_k^Tq_i & \text{if } i < k,\\
    \norm{v} & \text{if } i = k.
\end{cases} \]
Thus we have construct $R$ as well. Note that at each step we add a new column of $R$.

The QR algorithm is implemented on any scientific programming language you might use (R, python, julia, etc). You do not need to use Gram-Schmidt by hand. Indeed Gram-Schmidt is rarely used because it is numerical unstable and so other methods are used. Gram-Schmidt is good for the theoretical justification that the QR decomposition exists. 


\subsection{Eigen-decompositions}
Let $A \in \R^{n \times n}$, a vector $v$ is an eigenvector with eigenvalue $\lambda$ if $v \neq 0$ and $Av = \lambda v$. Most of our analysis will involve symmetric matrices (those that satisfy $A^T = A$). This simplfies the analysis of eigenvalues a lot.
\begin{thrm}
    \emph{[The Spectral Theorem]} If $A$ is a symmetric matirx, then $A$ has an orthonormal set of eigenvalues $v_1,\ldots,v_n$ with real eigenvalue $\la_1 \ge \la_2 \ge \ldots \ge \la_n$. That is 
    \[Av_i = \la_i v_i \text{ for } i = 1,\ldots,n, \]
    and for $i,j =1,\ldots,n$,
    \[v_i^Tv_j = \begin{cases}
        1 & \text{if } i =j,\\
        0 & \text{else}.
    \end{cases} \]
\end{thrm}
If we set $\Lambda = \text{diag}(\la_1,\ldots,\la_n)$ and $V = [v_1,v_2,\ldots,v_n]$, then the spectral theorem states that $A = V\Lambda V^T$ and $V^TV = I_n$. To see why this holds, note that $A = V \Lambda V^T$ if and only if $AV = V\Lambda$. By matrix multiplication $AV = [Av_1,\ldots, Av_n]$ and $V\Lambda = [\la_1v_1,\ldots \la_n v_n]$. Thus we have $AV =V\Lambda$ if and only if $Av_i = \lambda_i$ for each $i$. We also previously saw that the vector $\{v_i\}_{i=1}^n$ are orthonormal if and only if $V^TV = I_n$. 

A proof of the spectral theorem will be given in the document that is to be put on the course webpage. The spectral theorem makes calculations easier since $A^k = V\Lambda^k V^T$ for $k = 1,2,\ldots$ and if $A$ is invertible $A^{-k} = V\Lambda^{-k}V^T$. 
\subsection{Singular value decomposition}
The spectral theorem assumes that $A$ is symmetric. The following decomposition works for all matrices. Each matrix $A \in \R^{m \times n}$ with $m \ge n$ ($A$ is tall) has an SVD (singular value decomposition) into
\[A=U\Sigma V^T, \]
where $U \in \R^{m \times n}$, $\Sigma \in \R^{n \times n}$ is diagonal with diagonal entries $\sigma_1 \ge \sigma_2 \ge \ldots \ge \sigma_n \ge 0$, $V \in \R^{n \times n}$ and $U^TU = I_n = V^TV$. This decomposition tells us how $A$ acts on any vector $x \in \R^n$. 

Given $x \in \R^n$, we first change it to the basis of $V$, then we multiply by gains $\sigma_1,\ldots, \sigma_n$ and then give the ouput in terms of the $U$ basis. This is what the equation $Ax = U\Sigma V^Tx$ tells us. 

Note that $A^{-1} = V \Sigma^{-1}U^T$ if $\sigma_i > 0$ for all $i$ and $m=n$. Again see notes for the construction of the SVD. 
\section{Optimisation}
We didn't have time to cover this today. 
\end{document} 