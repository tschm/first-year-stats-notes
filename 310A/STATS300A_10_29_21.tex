\documentclass{article}
\usepackage{ae,aecompl}
\usepackage{todonotes}
\usepackage{chngcntr}
\usepackage{tikz-cd}
\usepackage{graphicx}
\graphicspath{ {./images/}}
\usepackage[all,cmtip]{xy}
\usepackage{amsmath, amscd}
\usepackage{amsthm}
\usepackage{amssymb}
\usepackage{amsfonts}
\usepackage{bm}
\usepackage{qsymbols}
\usepackage{latexsym}
\usepackage{mathrsfs}
\usepackage{mathtools}
\usepackage{cite}
\usepackage{color}
\usepackage{url}
\usepackage{enumerate}
\usepackage{verbatim}
\usepackage[draft=false, colorlinks=true]{hyperref}
\usepackage{pdfpages}
\usepackage[margin=1.2in]{geometry}
\usepackage{IEEEtrantools}
\usepackage{multirow}
\usepackage{fancyhdr}


\usepackage[nameinlink]{cleveref}


\DeclareMathOperator*{\ac}{accept}
\DeclareMathOperator*{\amax}{argmax}
\DeclareMathOperator*{\amin}{argmin}
\DeclareMathOperator*{\Aut}{Aut}
\newcommand {\al}{{\alpha}}
\newcommand {\abs}[1]{{\left\lvert#1\right\rvert}}
\newcommand {\A}{{\mathcal{A}}}
\newcommand {\AM}{{\mathrm{AM}}}
\newcommand {\AMp}{{\AM_{p}^{X}\!(\Ri_\w)}}
\newcommand {\B}{{\mathcal{B}}}
\DeclareMathOperator*{\Be}{Bern}
\newcommand {\Br}{{\dot{B}}}
\newcommand {\Ba}{{\mathfrak{B}}}
\newcommand {\C}{{\mathbb C}}
\newcommand {\ce}{\mathrm{c}}
\newcommand {\Ce}{\mathrm{C}}
\newcommand {\Cc}{\mathrm{C_{c}}}
\newcommand {\Ccinf}{\mathrm{C_{c}^{\infty}}}
\DeclareMathOperator{\DEV}{DEV}
\DeclareMathOperator{\diag}{diag}
\newcommand {\Di}{{\mathbb D}}
\newcommand {\dom}{\mathrm{dom}}
\newcommand {\ud}{\mathrm{d}}
\newcommand {\ue}{\mathrm{e}}
\newcommand {\eps}{\varepsilon}
\newcommand {\veps}{\varepsilon}
\newcommand {\vrho}{{\varrho}}
\newcommand {\E}{{\mathbb{E}}}
\newcommand {\Ec}{{\mathcal{E}}}
\newcommand {\Ell}{L}
\newcommand {\Ellp}{{L_{p}[0,1]}}
\newcommand {\Ellpprime}{{L_{p'}([0,1])}}
\newcommand {\Ellq}{{L_{q}([0,1])}}
\newcommand {\Ellqprime}{{L_{q'}([0,1])}}
\newcommand {\Ellr}{L^{r}}
\newcommand {\Ellone}{{L_{1}([0,1])}}
\newcommand {\Elltwo}{{L_{2}([0,1])}}
\newcommand {\Ellinfty}{L^{\infty}}
\newcommand {\Ellinftyc}{L_{\mathrm{c}}^{\infty}}
\newcommand {\exb}[1]{\exp\left\{#1\right\}}
\DeclareMathOperator*{\Exp}{Exp}
\DeclareMathOperator*{\Ext}{Ext}
\newcommand {\F}{{\mathcal{F}}}
\newcommand{\Fe}{{\mathbb{F}}}
\newcommand
{\G}{{\mathcal{G}}}
\newcommand {\HF}{\mathcal{H}_{\text{FIO}}^{1}(\Rd)}
\newcommand {\Hr}{H}
\newcommand {\HT}{\mathcal{H}}
\newcommand {\ui}{\mathrm{i}}
\newcommand {\I}{{I}}
\newcommand {\J}{{\mathcal{J}}}
\newcommand {\id}{{\mathrm{id}}}
\newcommand {\iid}{\stackrel{\mathclap{\normalfont\mbox{iid}}}{\sim}}
\newcommand {\im}{{\text{im }}}
\newcommand {\ind}{{\perp\!\!\!\perp}}
\newcommand{\indep}{\stackrel{\text{Indep}}{\sim}}
\DeclareMathOperator*{\Int}{int}
\newcommand {\intx}{{\overline{\int_{X}}}}
\newcommand {\inte}{{\overline{\int_{\E}}}}
\newcommand {\la}{\lambda}
\newcommand {\rb}{\rangle}
\newcommand {\lb}{{\langle}}
\newcommand {\La}{\Lambda}
\newcommand {\calL}{{\mathcal{L}}}
\newcommand {\lp}{{\mathcal{L}}^{p}}
\newcommand {\lpo}{{\overline{\mathcal{L}}^{p}\!}}
\newcommand {\Lpo}{{\overline{\Ell}^{p}\!}}
\newcommand {\M}{{\mathbf{M}}}
\newcommand {\Ma}{{\mathcal{M}}}
\newcommand {\N}{{{\mathbb N}}}
\newcommand {\Na}{{{\mathcal{N}}}}
\newcommand {\norm}[1]{\left\|#1\right\|}
\newcommand {\normm}[1]{{\left\vert\kern-0.25ex\left\vert\kern-0.25ex\left\vert #1 
    \right\vert\kern-0.25ex\right\vert\kern-0.25ex\right\vert}}
\newcommand {\Om}{{{\Omega}}}
\newcommand {\one}{{{\bf 1}}}
\newcommand {\pic}{\text{Pic }}
\newcommand {\ph}{{\varphi}}
\newcommand {\Pa}{{\mathbb{P}}}
\newcommand {\Po}{{\mathcal{P}}}
\newcommand {\Q}{{\mathbb{Q}}}
\newcommand {\R}{{\mathbb R}}
\newcommand {\Rd}{{\mathbb{R}^{d}}}
\DeclareMathOperator{\rej}{reject }
\newcommand {\Rn}{{\mathbb{R}^{n}}}
\newcommand {\cR}{{\mathcal{R}}}
\newcommand {\Rad}{{\mathrm{Rad}}}
\newcommand {\ran}{{\mathrm{ran}}}
\newcommand {\Ri}{{\mathrm{R}}}
\newcommand {\supp}{{\mathrm{supp}}}
\newcommand {\Se}{\mathrm{S}}
\newcommand {\Sp}{S^{*}(\Rn)}
\newcommand {\St}{{\mathrm{St}}}
\newcommand {\Sw}{\mathcal{S}}
\newcommand {\T}{{\mathcal{T}}}
\newcommand {\ta}{{\theta}}
\newcommand {\Ta}{{\Theta}}
\DeclareMathOperator {\V}{Var}
\newcommand {\w}{{\omega}}
\newcommand {\W}{{\mathrm{W}}}
\newcommand {\Wnp}{\text{$\mathrm{W}$\textsuperscript{$n,\!p$}}}
\newcommand {\Wnpeq}{\text{$\mathrm{W}$\textsuperscript{$n\!,\!p$}}}
\newcommand {\Wonep}{\text{$\mathrm{W}$\textsuperscript{$1,\!p$}}}
\newcommand {\Wonepeq}{\text{$\mathrm{W}$\textsuperscript{$1\!,\!p$}}}
\newcommand {\X}{{\mathcal{X}}}
\newcommand {\Z}{{{\mathbb Z}}}
\newcommand {\Za}{{\mathcal{Z}}}
\newcommand {\Zd}{{\Z[\sqrt{d}]}}
\newcommand {\vanish}[1]{\relax}

\newcommand {\wh}{\widehat}
\newcommand {\wt}{\widetilde}
\newcommand {\red}{\color{red}}

% Distributions
\newcommand{\normal}{\mathsf{N}}
\newcommand{\poi}{\mathsf{Poisson}}
\newcommand{\bern}{\mathsf{Bernoulli}}
\newcommand{\bin}{\mathsf{Binomal}}
\newcommand{\multi}{\mathsf{Multinomial}}




% put your command and environment definitions here




% some theorem environments
% remove "[theorem]" if you do not want them to use the same number sequence


  \newtheorem{thrm}{Theorem}
  \newtheorem{lemma}{Lemma}
  \newtheorem{prop}{Proposition}
  \newtheorem{cor}{Corollary}

  \newtheorem{conj}{Conjecture}
  \renewcommand{\theconj}{\Alph{conj}}  % numbered A, B, C etc

  \theoremstyle{definition}
  \newtheorem{defn}{Definition}
  \newtheorem{ex}{Example}
  \newtheorem{exs}{Examples}
  \newtheorem{question}{Question}
  \newtheorem{remark}{Remark}
  \newtheorem{notn}{Notation}
  \newtheorem{exer}{Exercise}




\title{STATS310A - Lecture 11}
\author{Persi Diaconis\\ Scribed by Michael Howes}
\date{10/26/21}

\pagestyle{fancy}
\fancyhf{}
\rhead{STATS310A - Lecture 11}
\lhead{10/26/21}
\rfoot{Page \thepage}

\begin{document}
\maketitle
\tableofcontents
\section{Expectation}
\begin{defn}
    Let $(\Om,\F,\Pa)$ be a probability space and suppose $X: \Om \to \R$ is measurable. We define \emph{the expectation of $X$} to be 
    \[ \E[X] = \int X(\w)\Pa(d\w),\]
    whenever the above intergal exists.
\end{defn}
Note that if $X=\delta_A$, then $\E[X] = \Pa(A)$. Thus we can once again reframe our fundamental question of probability:
\begin{center}
    Given a random variable $X$, compute or approximate $\E[X]$.
\end{center}
which generalizes our previous versions of this question.
\subsection{Sums of random variables}
Let $(\Om,\F,\Pa)$ be a probability space and suppose $X,Y$ are independent random variables on $\Om$. Define the following probabilities on $\R$,
\[\mu(A) = \Pa(X \in A) \text{ and } \nu(A) = \Pa(Y\in A). \]
Since $X,Y$ are independent, we know that for all measurable $C \subseteq \R^2$,
\[\Pa((X,Y) \in C) = (\mu \times \nu)(C), \]
where $\mu \times \nu$ is the product measure given by
\[(\mu \times \nu)(C) = \int_\R \mu(C_y)\nu(dy) = \int_\R \nu(C_x)\mu(dx). \]
Given $D \subseteq \R$ measurable, define $C = \{(x,y) : x+y \in D\} \subseteq \R^2$. Note that $C_y = D-y$ and $C_x = D-x$. Thus
\begin{align*}
    \Pa(X+Y \in D) &= \Pa((X,Y) \in C)\\
    &= \int_\R \nu(C_x)\mu(dx)\\
    &= \int_\R \nu(D-x)\mu(dx).
\end{align*}
And likewise
\[\Pa(X+Y \in D) = \int_\R \mu(D-y)\nu(dy). \]
The above defines a measure on $\R$ and is called the convolution of $\mu$ and $\nu$. It is denoted $\mu * \nu$.
\begin{ex}
    Suppose $\mu = \text{Poission}(\ta)$ and $\nu = \text{Possion}(\eta)$ for some $\ta,\eta \ge 0$. That is\
    \[\mu(A) = \sum_{j \in A}\frac{e^{-\ta}\ta^j}{j!} \text{ and } \nu(B) = \sum_{j \in B} \frac{e^{-\eta}\eta^j}{j!},\]
    where $A,B \subseteq \{0,1,2,\ldots\}$. Note that for $l = 0,1,2,\ldots$,
    \begin{align*}
        (\mu *\nu)(\{l\}) &= \sum_{a=0}^l \frac{e^{-\ta}\ta^a}{a!}\frac{e^{-\eta}\eta^{l-a}}{(l-a)!}\\
        &= \frac{e^{-(\ta+\eta)}\eta^l}{l!} \sum_{a=0}^l \binom{l}{a} \left(\frac{\ta}{\eta}\right)^a\\
        &=\frac{e^{-(\ta+\eta)}\eta^l}{l!}\left(1+\frac{\ta}{\eta}\right)^l\\
        &=\frac{e^{-(\ta+\eta)}(\ta+\eta)^l}{l!}.
    \end{align*}
    Thus $\mu *\nu = \text{Poission}(\ta+\eta)$.
\end{ex}
\begin{ex}
    Similarly if $X \sim \Na(\ta_1,\sigma^2_1)$ and $Y \sim \Na(\ta_2,\sigma^2_2)$ and $X,Y$ are independent, then $X+Y \sim \Na(\ta_1+\ta_2,\sigma_1^2+\sigma_2^2)$.
\end{ex}
\section{Homework}
Read chapters 21-22 and do problem 20.8, 20.9, 21.3 and 21.6. There are two other problems which we describe now.
\subsection{Problem (A)}
This problem is about the $\beta-\Gamma$ calculus. For $\al >0$, we say $X \sim \text{Gamma}(\al)$ on $[0,\infty)$ if $X$ has density 
\[\frac{e^{-x}x^{\al-1}}{\Gamma(\al)}, \text{ for } x \ge 0. \]
For $\al,\beta > 0$, we say that $W \sim \text{Beta}(\al,\beta)$ on $[0,1]$ is $W$ has density 
\[\frac{\Gamma(\al+\beta)}{\Gamma(\al)\Gamma(\beta)} x^{\al-1}(1-x)^{\beta-1}, \text{ for } 0\le x \le 1. \]
On the homework we will show:
\begin{enumerate}
    \item If $X \sim \text{Gamma}(\al)$ and $Y \sim \text{Gamma}(\beta)$ and $X$ is independent of $Y$, then $\frac{X}{X+Y}$ and $X+Y$ are independent, $\frac{X}{X+Y} \sim \text{Beta}(\al,\beta)$ and $X+Y \sim \text{Gamma}(\al+\beta)$.
    \item If $X,Y,Z \sim \text{Gamma}(\al),\text{Gamma}(\beta), \text{Gamma}(\gamma)$ are independent, then 
    \[\frac{X}{X+Y}, \frac{X+Y}{X+Y+Z} \text{ and } X+Y+Z,\]
    are independent and they have distributions $\text{Beta}(\al,\beta)$, $\text{Beta}(\al+\beta,\gamma)$ and $\text{Gamma}(\al+\beta+\gamma)$ respectively.
\end{enumerate}
\subsection{Problem (B)}
Before discussing problem (B) on the homework. We will state, prove and apply a theorem.
\begin{thrm}
    Suppose $X \ge 0$, then 
    \[\E[X] = \int_0^\infty \Pa(X \ge t)dt = \int_0^\infty \Pa(X >t)dt. \]
\end{thrm}
Note that the two integrals are indeed equal since $\Pa(X=t) >0$ for at most countably many $t$. Thus the functions $\Pa(X\ge t)$ and $\Pa(X>t)$ agree almost everywhere with respect to Lesbegue measure. 
\begin{proof}
    Suppose that $X = \sum_{j=1}^k x_j \delta_{A_j}$ with $0 \le x_1\le x_2\le \ldots\le x_k$. Then 
    \begin{align*}
        \E[X] &=\sum_{j=1}^k x_j \Pa(X=x_j)\\
        &=\sum_{j=1}^{k-1}x_j\left(\Pa(X\ge x_j)-\Pa(X>x_{j+1})\right)+x_k\Pa(X\ge x_k)\\
        &=x_1\Pa(X \ge x_1)+\sum_{j=1}^k(x_j-x_{j-1})\Pa(X \ge x_j).
    \end{align*}
    Observe $\Pa(X \ge t) = \Pa(X \ge x_1) = 1$ for $t \in [0,x_1]$ and $\Pa(X \ge t) = 0$ for $t > x_k$. Also note that ff $x_{j-1} < t < x_j$, then 
    \[\Pa(X \ge t) = \Pa(X \ge x_j). \]
    Thus the function $t \mapsto \Pa(X \ge t)$ is a step function and we have 
    \[\int_0^\infty \Pa(X \ge t)dt = x_1+\Pa(X\ge x_i)+\sum_{j=1}^k x_j-x_{j-1})\Pa(X \ge x_j) = \E[X]. \]
    A limiting argument via the monotone convergence theorem allows us to conclude that the result holds for all $X \ge 0$. 
\end{proof}
\end{document}