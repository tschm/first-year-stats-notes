\documentclass{article}
\usepackage{ae,aecompl}
\usepackage{todonotes}
\usepackage{chngcntr}
\usepackage{tikz-cd}
\usepackage{graphicx}
\graphicspath{ {./images/}}
\usepackage[all,cmtip]{xy}
\usepackage{amsmath, amscd}
\usepackage{amsthm}
\usepackage{amssymb}
\usepackage{amsfonts}
\usepackage{bm}
\usepackage{qsymbols}
\usepackage{latexsym}
\usepackage{mathrsfs}
\usepackage{mathtools}
\usepackage{cite}
\usepackage{color}
\usepackage{url}
\usepackage{enumerate}
\usepackage{verbatim}
\usepackage[draft=false, colorlinks=true]{hyperref}
\usepackage{pdfpages}
\usepackage[margin=1.2in]{geometry}
\usepackage{IEEEtrantools}
\usepackage{multirow}
\usepackage{fancyhdr}


\usepackage[nameinlink]{cleveref}


\DeclareMathOperator*{\ac}{accept}
\DeclareMathOperator*{\amax}{argmax}
\DeclareMathOperator*{\amin}{argmin}
\DeclareMathOperator*{\Aut}{Aut}
\newcommand {\al}{{\alpha}}
\newcommand {\abs}[1]{{\left\lvert#1\right\rvert}}
\newcommand {\A}{{\mathcal{A}}}
\newcommand {\AM}{{\mathrm{AM}}}
\newcommand {\AMp}{{\AM_{p}^{X}\!(\Ri_\w)}}
\newcommand {\B}{{\mathcal{B}}}
\DeclareMathOperator*{\Be}{Bern}
\newcommand {\Br}{{\dot{B}}}
\newcommand {\Ba}{{\mathfrak{B}}}
\newcommand {\C}{{\mathbb C}}
\newcommand {\ce}{\mathrm{c}}
\newcommand {\Ce}{\mathrm{C}}
\newcommand {\Cc}{\mathrm{C_{c}}}
\newcommand {\Ccinf}{\mathrm{C_{c}^{\infty}}}
\DeclareMathOperator{\DEV}{DEV}
\DeclareMathOperator{\diag}{diag}
\newcommand {\Di}{{\mathbb D}}
\newcommand {\dom}{\mathrm{dom}}
\newcommand {\ud}{\mathrm{d}}
\newcommand {\ue}{\mathrm{e}}
\newcommand {\eps}{\varepsilon}
\newcommand {\veps}{\varepsilon}
\newcommand {\vrho}{{\varrho}}
\newcommand {\E}{{\mathbb{E}}}
\newcommand {\Ec}{{\mathcal{E}}}
\newcommand {\Ell}{L}
\newcommand {\Ellp}{{L_{p}[0,1]}}
\newcommand {\Ellpprime}{{L_{p'}([0,1])}}
\newcommand {\Ellq}{{L_{q}([0,1])}}
\newcommand {\Ellqprime}{{L_{q'}([0,1])}}
\newcommand {\Ellr}{L^{r}}
\newcommand {\Ellone}{{L_{1}([0,1])}}
\newcommand {\Elltwo}{{L_{2}([0,1])}}
\newcommand {\Ellinfty}{L^{\infty}}
\newcommand {\Ellinftyc}{L_{\mathrm{c}}^{\infty}}
\newcommand {\exb}[1]{\exp\left\{#1\right\}}
\DeclareMathOperator*{\Exp}{Exp}
\DeclareMathOperator*{\Ext}{Ext}
\newcommand {\F}{{\mathcal{F}}}
\newcommand{\Fe}{{\mathbb{F}}}
\newcommand
{\G}{{\mathcal{G}}}
\newcommand {\HF}{\mathcal{H}_{\text{FIO}}^{1}(\Rd)}
\newcommand {\Hr}{H}
\newcommand {\HT}{\mathcal{H}}
\newcommand {\ui}{\mathrm{i}}
\newcommand {\I}{{I}}
\newcommand {\J}{{\mathcal{J}}}
\newcommand {\id}{{\mathrm{id}}}
\newcommand {\iid}{\stackrel{\mathclap{\normalfont\mbox{iid}}}{\sim}}
\newcommand {\im}{{\text{im }}}
\newcommand {\ind}{{\perp\!\!\!\perp}}
\newcommand{\indep}{\stackrel{\text{Indep}}{\sim}}
\DeclareMathOperator*{\Int}{int}
\newcommand {\intx}{{\overline{\int_{X}}}}
\newcommand {\inte}{{\overline{\int_{\E}}}}
\newcommand {\la}{\lambda}
\newcommand {\rb}{\rangle}
\newcommand {\lb}{{\langle}}
\newcommand {\La}{\Lambda}
\newcommand {\calL}{{\mathcal{L}}}
\newcommand {\lp}{{\mathcal{L}}^{p}}
\newcommand {\lpo}{{\overline{\mathcal{L}}^{p}\!}}
\newcommand {\Lpo}{{\overline{\Ell}^{p}\!}}
\newcommand {\M}{{\mathbf{M}}}
\newcommand {\Ma}{{\mathcal{M}}}
\newcommand {\N}{{{\mathbb N}}}
\newcommand {\Na}{{{\mathcal{N}}}}
\newcommand {\norm}[1]{\left\|#1\right\|}
\newcommand {\normm}[1]{{\left\vert\kern-0.25ex\left\vert\kern-0.25ex\left\vert #1 
    \right\vert\kern-0.25ex\right\vert\kern-0.25ex\right\vert}}
\newcommand {\Om}{{{\Omega}}}
\newcommand {\one}{{{\bf 1}}}
\newcommand {\pic}{\text{Pic }}
\newcommand {\ph}{{\varphi}}
\newcommand {\Pa}{{\mathbb{P}}}
\newcommand {\Po}{{\mathcal{P}}}
\newcommand {\Q}{{\mathbb{Q}}}
\newcommand {\R}{{\mathbb R}}
\newcommand {\Rd}{{\mathbb{R}^{d}}}
\DeclareMathOperator{\rej}{reject }
\newcommand {\Rn}{{\mathbb{R}^{n}}}
\newcommand {\cR}{{\mathcal{R}}}
\newcommand {\Rad}{{\mathrm{Rad}}}
\newcommand {\ran}{{\mathrm{ran}}}
\newcommand {\Ri}{{\mathrm{R}}}
\newcommand {\supp}{{\mathrm{supp}}}
\newcommand {\Se}{\mathrm{S}}
\newcommand {\Sp}{S^{*}(\Rn)}
\newcommand {\St}{{\mathrm{St}}}
\newcommand {\Sw}{\mathcal{S}}
\newcommand {\T}{{\mathcal{T}}}
\newcommand {\ta}{{\theta}}
\newcommand {\Ta}{{\Theta}}
\DeclareMathOperator {\V}{Var}
\newcommand {\w}{{\omega}}
\newcommand {\W}{{\mathrm{W}}}
\newcommand {\Wnp}{\text{$\mathrm{W}$\textsuperscript{$n,\!p$}}}
\newcommand {\Wnpeq}{\text{$\mathrm{W}$\textsuperscript{$n\!,\!p$}}}
\newcommand {\Wonep}{\text{$\mathrm{W}$\textsuperscript{$1,\!p$}}}
\newcommand {\Wonepeq}{\text{$\mathrm{W}$\textsuperscript{$1\!,\!p$}}}
\newcommand {\X}{{\mathcal{X}}}
\newcommand {\Z}{{{\mathbb Z}}}
\newcommand {\Za}{{\mathcal{Z}}}
\newcommand {\Zd}{{\Z[\sqrt{d}]}}
\newcommand {\vanish}[1]{\relax}

\newcommand {\wh}{\widehat}
\newcommand {\wt}{\widetilde}
\newcommand {\red}{\color{red}}

% Distributions
\newcommand{\normal}{\mathsf{N}}
\newcommand{\poi}{\mathsf{Poisson}}
\newcommand{\bern}{\mathsf{Bernoulli}}
\newcommand{\bin}{\mathsf{Binomal}}
\newcommand{\multi}{\mathsf{Multinomial}}




% put your command and environment definitions here




% some theorem environments
% remove "[theorem]" if you do not want them to use the same number sequence


  \newtheorem{thrm}{Theorem}
  \newtheorem{lemma}{Lemma}
  \newtheorem{prop}{Proposition}
  \newtheorem{cor}{Corollary}

  \newtheorem{conj}{Conjecture}
  \renewcommand{\theconj}{\Alph{conj}}  % numbered A, B, C etc

  \theoremstyle{definition}
  \newtheorem{defn}{Definition}
  \newtheorem{ex}{Example}
  \newtheorem{exs}{Examples}
  \newtheorem{question}{Question}
  \newtheorem{remark}{Remark}
  \newtheorem{notn}{Notation}
  \newtheorem{exer}{Exercise}




\title{STATS300B -- Lecture 7}
\author{Julia Palacios\\ Scribed by Michael Howes}
\date{01/25/22}

\pagestyle{fancy}
\fancyhf{}
\rhead{STATS300B -- Lecture7 }
\lhead{01/25/22}
\rfoot{Page \thepage}

\begin{document}
\maketitle
\tableofcontents
\section{Motivation}
\begin{enumerate}
    \item We want to prove consistency of the MLE.
    \item The MLE maximizes the log-likelihood which is an empirical average. In particular, the log-likelihood is a random function.
    \item We will thus use the weak law for random functions.
\end{enumerate}
\section{Random functions}
Recall the following,
\begin{theorem}
    Let $K$ be a compact set and suppose $X_1,X_2,\ldots$ are i.i.d. and $W_i(t)=h(t,X_i) \in C(K)$ for all values of $X_i$. Let $\mu(t) = \E[W_1(t)]$ and assume that $\E[\norm{W}_\infty] < \infty$, then $\norm{\bar{W}_n -\mu}_\infty \topp 0$.
\end{theorem}
We will also use the following theorem about the optimizers of random functions.
\begin{theorem}
    Let $G_n$ be random functions in $C(K)$ where $K$ is compact. Let $g \in C(K)$ be a deterministic function. Suppose that $\norm{G_n - g}_\infty \topp 0$, then
    \begin{enumerate}
        \item If $t_n \topp t^*$, then $G_n(t_n)\topp g(t^*)$.
        \item Let $t_n$ be a random variable maximizing $G_n$. If $g$ achieves its maximum at a unique value $t^*$, then $t_n \topp t^*$.
        \item Suppose that $K \subseteq \R$ and $t_n$ is a random variable solving $G_n(t_n)=0$. If $t^*$ is the unique value in $K$ such that $g(t^*)=0$, then $t_n \topp t^*$.
    \end{enumerate}
\end{theorem}
\section{Kullback--Leibler information}
\begin{definition}
    Let $P$ and $Q$ be probability measures with densities $p$ and $q$ with respect to a common $\sigma$-finite measure $\mu$. The \emph{Kullback--Leibler information} of $P$ and $Q$ is defined to be,
    \[K(P,Q) = \E_P\left[
        \log\left(\frac{p(X)}{q(X)}\right)
    \right]. \]
\end{definition}
\begin{proposition}
    For all probability distributions, $K(P,Q) \ge 0$ and $K(P,Q)=0$ if and only if $P=Q$.
\end{proposition}
\begin{proof}
    Recall that $-\log$ is convex. Thus, by Jensen's inequality,
    \begin{align*}
        \E_P\left[\log\left(\frac{p(X)}{q(X)}\right)\right]&=\E_P\left[-\log\left(\frac{q(X)}{p(X)}\right)\right]\\
        &\ge -\log\left(\E_P\left[\frac{q(X)}{p(X)}\right]\right)\\
        &=-\log\left(\int_{p(x)\neq 0}q(x)dx\right)\\
        &\ge -\log(1)\\
        &=0.
    \end{align*}
    Furthermore, we have equality when $\frac{q(X)}{p(X)}$ is constant $P$-almost everywhere. But since $p$ and $q$ are densities this happens only when $P=Q$.
\end{proof}
Kullback--Leibler distant relates to the log-likelihood in the following way,
\begin{lemma}
    Consider a model $\Po = \{P_\ta : \ta \in \Om\}$ and consider the following assumptions,
    \begin{enumerate}
        \item If $\ta \neq \ta^*$, then $P_\ta \neq P_{\ta^*}$.
        \item There exists a measure $\mu$ such that for every $\ta$, $P_\ta$ has a density $p_\ta$ with respect to $\mu$.
        \item The support of $P_\ta$, $\{x : p_\ta(x)>0\}$ does not depend on $\ta$.
    \end{enumerate}
    Then, if $X_1,\ldots, X_n$ are i.i.d. $P_{\ta^*}$, then 
    \[\frac{1}{n}\left[\frac{L_n(\ta^*)}{L_n(\ta)}\right] \toas K(P_{\ta^*},P_\ta),\]
    where $L_n(\ta)$ is the log-likelihood evaluated at $\ta$.  In particular, for all $\ta\neq \ta^*$,
    \[\Pa(L_n(\ta^*|X) \ge L_n(\ta|X)) \to 1.\] 
\end{lemma}
\begin{proof}
    The proof is a simple application of the strong law of large numbers.
\end{proof}
We are now ready to state the consistency theorem for the MLE. 
\begin{theorem}
    Suppose that the three assumption of the previous lemma hold and that $\Om$ is compact. Fix $\ta^* \in \Om$ and define $h:\Om \times \X \to \R$ as the function,
    \[h(\ta,x) = \log\left[\frac{p_\ta(X)}{p_{\ta^*}(x)}\right]. \]
    Let $X_0,X_1,X_2,\ldots$ be i.i.d. samples from $P_{\ta^*}$ and define $W_i(\ta) = h(\ta,X_i)$. Suppose that $h$ is continuous in $\ta$ for every $x$ and,
    \[\E_{\ta^*}[\norm{W_0}_\infty] < \infty. \]
    Define $\wh{\ta}_n$ to be the MLE given $X_1,\ldots,X_n$. Then, under $P_{\ta^*}$, $\wh{\ta}_n \topp \ta$.
\end{theorem}

\section{Asymptotic normality}
We now know that the MLE is consistent, but what can we say about the limiting distribution? 
\begin{theorem}
    Let $\Po = \{P_\ta:\ta \in \Om\}$ be a model for $X \in \X$. Suppose that $\Om \subseteq \R$ and that $P_\ta$ has a density $p_\ta$ with respect to a common base measure $\mu$. Suppose that the following hold,
    \begin{enumerate}
        \item The support of $p_\ta$ does not depend on $\ta$.
        \item For every $x \in \X$, $\frac{\partial^2}{\partial \ta^2} p_\ta(x)$ exists and is continuous in $\ta$.
        \item If $l(\ta) = \log p_\ta(x)$, then the Fisher information exists, is finite and can be calculated as either
        \[I(\ta) = \E_\ta\left[\left(\frac{\partial}{\partial \ta}l(\ta)\right)^2\right]~~\text{ or }~~I(\ta) = -\E_\ta\left[\frac{\partial^2}{\partial \ta^2}l(\ta)\right],\]
        and $\E\left[\frac{\partial}{\partial \ta}l(\ta)\right]=0$.
        \item For every $\ta$ in the interior of $\Om$, there exists $\eps > 0$ such that $\E_\ta \norm{\one_{[\ta-\eps,\ta+\eps]}\frac{\partial^2}{\partial\ta^2}l(\ta)}< \infty$.
        \item The MLE $\wh{\ta}_n$ is consistent.
    \end{enumerate}
    Then for any $\ta^*$ in the interior of $\Om$, if $X_i \iid P_{\ta^*}$, then 
    \[\sqrt{n}\left(\wh{\ta}_n-\ta\right) \todd \normal\left(0, \frac{1}{I(\ta)}\right).\]
\end{theorem}
\end{document}