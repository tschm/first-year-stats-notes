\documentclass{article}
\usepackage{ae,aecompl}
\usepackage{todonotes}
\usepackage{chngcntr}
\usepackage{tikz-cd}
\usepackage{graphicx}
\graphicspath{ {./images/}}
\usepackage[all,cmtip]{xy}
\usepackage{amsmath, amscd}
\usepackage{amsthm}
\usepackage{amssymb}
\usepackage{amsfonts}
\usepackage{bm}
\usepackage{qsymbols}
\usepackage{latexsym}
\usepackage{mathrsfs}
\usepackage{mathtools}
\usepackage{cite}
\usepackage{color}
\usepackage{url}
\usepackage{enumerate}
\usepackage{verbatim}
\usepackage[draft=false, colorlinks=true]{hyperref}
\usepackage{pdfpages}
\usepackage[margin=1.2in]{geometry}
\usepackage{IEEEtrantools}
\usepackage{multirow}
\usepackage{fancyhdr}


\usepackage[nameinlink]{cleveref}


\DeclareMathOperator*{\ac}{accept}
\DeclareMathOperator*{\amax}{argmax}
\DeclareMathOperator*{\amin}{argmin}
\DeclareMathOperator*{\Aut}{Aut}
\newcommand {\al}{{\alpha}}
\newcommand {\abs}[1]{{\left\lvert#1\right\rvert}}
\newcommand {\A}{{\mathcal{A}}}
\newcommand {\AM}{{\mathrm{AM}}}
\newcommand {\AMp}{{\AM_{p}^{X}\!(\Ri_\w)}}
\newcommand {\B}{{\mathcal{B}}}
\DeclareMathOperator*{\Be}{Bern}
\newcommand {\Br}{{\dot{B}}}
\newcommand {\Ba}{{\mathfrak{B}}}
\newcommand {\C}{{\mathbb C}}
\newcommand {\ce}{\mathrm{c}}
\newcommand {\Ce}{\mathrm{C}}
\newcommand {\Cc}{\mathrm{C_{c}}}
\newcommand {\Ccinf}{\mathrm{C_{c}^{\infty}}}
\DeclareMathOperator{\DEV}{DEV}
\DeclareMathOperator{\diag}{diag}
\newcommand {\Di}{{\mathbb D}}
\newcommand {\dom}{\mathrm{dom}}
\newcommand {\ud}{\mathrm{d}}
\newcommand {\ue}{\mathrm{e}}
\newcommand {\eps}{\varepsilon}
\newcommand {\veps}{\varepsilon}
\newcommand {\vrho}{{\varrho}}
\newcommand {\E}{{\mathbb{E}}}
\newcommand {\Ec}{{\mathcal{E}}}
\newcommand {\Ell}{L}
\newcommand {\Ellp}{{L_{p}[0,1]}}
\newcommand {\Ellpprime}{{L_{p'}([0,1])}}
\newcommand {\Ellq}{{L_{q}([0,1])}}
\newcommand {\Ellqprime}{{L_{q'}([0,1])}}
\newcommand {\Ellr}{L^{r}}
\newcommand {\Ellone}{{L_{1}([0,1])}}
\newcommand {\Elltwo}{{L_{2}([0,1])}}
\newcommand {\Ellinfty}{L^{\infty}}
\newcommand {\Ellinftyc}{L_{\mathrm{c}}^{\infty}}
\newcommand {\exb}[1]{\exp\left\{#1\right\}}
\DeclareMathOperator*{\Exp}{Exp}
\DeclareMathOperator*{\Ext}{Ext}
\newcommand {\F}{{\mathcal{F}}}
\newcommand{\Fe}{{\mathbb{F}}}
\newcommand
{\G}{{\mathcal{G}}}
\newcommand {\HF}{\mathcal{H}_{\text{FIO}}^{1}(\Rd)}
\newcommand {\Hr}{H}
\newcommand {\HT}{\mathcal{H}}
\newcommand {\ui}{\mathrm{i}}
\newcommand {\I}{{I}}
\newcommand {\J}{{\mathcal{J}}}
\newcommand {\id}{{\mathrm{id}}}
\newcommand {\iid}{\stackrel{\mathclap{\normalfont\mbox{iid}}}{\sim}}
\newcommand {\im}{{\text{im }}}
\newcommand {\ind}{{\perp\!\!\!\perp}}
\newcommand{\indep}{\stackrel{\text{Indep}}{\sim}}
\DeclareMathOperator*{\Int}{int}
\newcommand {\intx}{{\overline{\int_{X}}}}
\newcommand {\inte}{{\overline{\int_{\E}}}}
\newcommand {\la}{\lambda}
\newcommand {\rb}{\rangle}
\newcommand {\lb}{{\langle}}
\newcommand {\La}{\Lambda}
\newcommand {\calL}{{\mathcal{L}}}
\newcommand {\lp}{{\mathcal{L}}^{p}}
\newcommand {\lpo}{{\overline{\mathcal{L}}^{p}\!}}
\newcommand {\Lpo}{{\overline{\Ell}^{p}\!}}
\newcommand {\M}{{\mathbf{M}}}
\newcommand {\Ma}{{\mathcal{M}}}
\newcommand {\N}{{{\mathbb N}}}
\newcommand {\Na}{{{\mathcal{N}}}}
\newcommand {\norm}[1]{\left\|#1\right\|}
\newcommand {\normm}[1]{{\left\vert\kern-0.25ex\left\vert\kern-0.25ex\left\vert #1 
    \right\vert\kern-0.25ex\right\vert\kern-0.25ex\right\vert}}
\newcommand {\Om}{{{\Omega}}}
\newcommand {\one}{{{\bf 1}}}
\newcommand {\pic}{\text{Pic }}
\newcommand {\ph}{{\varphi}}
\newcommand {\Pa}{{\mathbb{P}}}
\newcommand {\Po}{{\mathcal{P}}}
\newcommand {\Q}{{\mathbb{Q}}}
\newcommand {\R}{{\mathbb R}}
\newcommand {\Rd}{{\mathbb{R}^{d}}}
\DeclareMathOperator{\rej}{reject }
\newcommand {\Rn}{{\mathbb{R}^{n}}}
\newcommand {\cR}{{\mathcal{R}}}
\newcommand {\Rad}{{\mathrm{Rad}}}
\newcommand {\ran}{{\mathrm{ran}}}
\newcommand {\Ri}{{\mathrm{R}}}
\newcommand {\supp}{{\mathrm{supp}}}
\newcommand {\Se}{\mathrm{S}}
\newcommand {\Sp}{S^{*}(\Rn)}
\newcommand {\St}{{\mathrm{St}}}
\newcommand {\Sw}{\mathcal{S}}
\newcommand {\T}{{\mathcal{T}}}
\newcommand {\ta}{{\theta}}
\newcommand {\Ta}{{\Theta}}
\DeclareMathOperator {\V}{Var}
\newcommand {\w}{{\omega}}
\newcommand {\W}{{\mathrm{W}}}
\newcommand {\Wnp}{\text{$\mathrm{W}$\textsuperscript{$n,\!p$}}}
\newcommand {\Wnpeq}{\text{$\mathrm{W}$\textsuperscript{$n\!,\!p$}}}
\newcommand {\Wonep}{\text{$\mathrm{W}$\textsuperscript{$1,\!p$}}}
\newcommand {\Wonepeq}{\text{$\mathrm{W}$\textsuperscript{$1\!,\!p$}}}
\newcommand {\X}{{\mathcal{X}}}
\newcommand {\Z}{{{\mathbb Z}}}
\newcommand {\Za}{{\mathcal{Z}}}
\newcommand {\Zd}{{\Z[\sqrt{d}]}}
\newcommand {\vanish}[1]{\relax}

\newcommand {\wh}{\widehat}
\newcommand {\wt}{\widetilde}
\newcommand {\red}{\color{red}}

% Distributions
\newcommand{\normal}{\mathsf{N}}
\newcommand{\poi}{\mathsf{Poisson}}
\newcommand{\bern}{\mathsf{Bernoulli}}
\newcommand{\bin}{\mathsf{Binomal}}
\newcommand{\multi}{\mathsf{Multinomial}}




% put your command and environment definitions here




% some theorem environments
% remove "[theorem]" if you do not want them to use the same number sequence


  \newtheorem{thrm}{Theorem}
  \newtheorem{lemma}{Lemma}
  \newtheorem{prop}{Proposition}
  \newtheorem{cor}{Corollary}

  \newtheorem{conj}{Conjecture}
  \renewcommand{\theconj}{\Alph{conj}}  % numbered A, B, C etc

  \theoremstyle{definition}
  \newtheorem{defn}{Definition}
  \newtheorem{ex}{Example}
  \newtheorem{exs}{Examples}
  \newtheorem{question}{Question}
  \newtheorem{remark}{Remark}
  \newtheorem{notn}{Notation}
  \newtheorem{exer}{Exercise}




\title{STATS300B -- Lecture 6}
\author{Julia Palacios\\ Scribed by Michael Howes}
\date{01/20/22}

\pagestyle{fancy}
\fancyhf{}
\rhead{STATS300B -- Lecture 6}
\lhead{01/20/22}
\rfoot{Page \thepage}

\begin{document}
\maketitle
\tableofcontents
\section{The delta method}
We ended last class with the statement of the higher order delta method. We stated the following.
\begin{theorem}[Delta method 3 (higher order)]
    Suppose that $X_n$ are random $k$-vectors such that 
    \[r_n(X_n-\ta) \todd X, \]
    where $r_n$ is a deterministic function with $r_n \to +\infty$. Let $\phi$ be a real-valued function that is twice differentiable at $\ta$ with $\phi'(\ta)=0$. Then,
    \[r_n^2(\phi(X_n)-\phi(\ta)) \todd \frac{1}{2}X^T\nabla^2\phi(\ta)T.\]
\end{theorem}
We will now prove the above.
\begin{proof}
    Note that,
    \begin{align*}
        \phi(X_n) &=\phi(\ta)+\nabla \phi(\ta)^T(X_n-\ta) + (X_n-\ta)^T\nabla^2 \phi(\ta)(T_n-\ta)+o(\norm{X_n-\ta}_2^2)\\
        &=\phi(\ta) + \frac{1}{2}(X_n-\ta)^T\nabla^2 \phi(\ta)(X_n-\ta)+o(\norm{X_n-\ta}_2^2).
    \end{align*}
\end{proof}
Note that $o(\norm{X_n-\ta}_2^2) = o_p(r_n^{-2})$. Thus, by Slutsky's
\begin{align*}
    r_n^2\left(\phi(X_n)-\phi(\ta)\right)&=\frac{1}{2}(r_n(X_n-\ta))^T\nabla^2 \phi(\ta)(r_n(X_n-\ta))+r_n^2o_p(r_n^{-2})\\
    &\todd  \frac{1}{2}X^T\nabla^2 \phi(\ta) X.\qedhere 
\end{align*}
The central limit theorem gives a special one-dimensional case of the higher-order delta method. If $X_1,\ldots$ are i.i.d. with mean $\mu$ and variance $\sigma^2 < \infty$, then $\sqrt{n}(\bar{X}_n-\mu) \todd \normal(0,\sigma^2)$. Thus, if $g$ is a twice differentiable at $\mu$, then 
\[n(g(\bar{X}_n)-g(\mu)) \todd g''(\mu)\sigma^2 Z^2, \]
where $Z \sim \normal(0,1)$. So that, $\frac{n}{g''(\mu)\sigma^2}(g(\bar{X}_n)-g(\mu)) \todd \chi^2_1$.

\section{Maximum likelihood estimation}
Suppose we have data $X$ in the form of $n$ i.i.d. observations $X_1,\ldots,X_n$ drawn from a distribution $\Pa_\ta$ with density $p_\ta$. The likelihood $L(\ta)= p_\ta(X)$ is the density evaluated at $X$ viewed as a function of $\ta$. The value $\wh{\ta}$ that maximizes $L(\ta)$ is called the maximum likelihood estimator (MLE) of $\ta$. Much of this course will focus on properties of MLEs.
\subsection{MLEs for one-dimensional exponential families}
Suppose that $X$ is generated from a one dimensional exponential family in canonical form. That is, 
\[p_\eta(x) = \exp\{\eta T(x) - A(\eta)\}h(x), \]
where $\eta$ is the natural parameter for the family, $T(X)$ are the sufficient statistics for the family and $A(\eta) = \log\left(\int \exb{\eta T(x)}h(x)dx\right)$ is the log-partition function for the family. The log-partition function is convex and differentiable  with,
\[A'(\eta) = \E_\eta[T(X)],\]
and 
\[A''(\eta) = \V_\eta(T(X)).\]
Another name of the $A(\eta)$ is the \emph{cummulant generating function of $T(X)$}. Suppose now that we have an i.i.d. sample of size $n$ drawn from $p_\eta$. Let $l(\eta) = \log(p_\eta(X_1,\ldots,X_n))$ be the log-likelihood function. Then,
\begin{align*}(
    l(\eta)&= \eta \sum_{i=1}^n T(X_i)-nA(\eta)+\log\left(\prod_{i=1}^n h(X_i)\right)\\
    \therefore l'(\eta) &= \sum_{i=1}^n T(X_i)-nA'(\eta) = \sum_{i=1}^n T(X_i)-n\E_\eta[T(X)]\\
    \therefore l''(\eta) &= -nA''(\eta) = -n\V_\eta(T(X)) \le 0.
\end{align*}
Thus, any solution to $l'(\eta)=0$ is a local maximum of $l$. It follows that the MLE $\wh{\eta}$ solves the equation,
\[\E_{\wh{\eta}}[T(X)] = \frac{1}{n}\sum_{i=1}^n T(X_i) = \bar{T}_n  \]
Thus, the MLE is a type of method of moments estimator in this example. In particular, we have $\wh{\eta} = \phi(\bar{T})$ where $\phi$ can be thought of as $(A')^{-1}$. By the central limit theorem,
\[\sqrt{n}(\bar{T}-\E_\eta[T(X)]) \todd \normal(0,\V_\eta(T(X))) = \normal(0,A''(\eta)). \]
Thus, by the delta method,
\[\sqrt{n}(\wh{\eta}-\eta) \todd \normal(0, \phi'(\eta)^2A''(\eta)). \]
We know that $\phi'(\ta) = \frac{1}{A''(\eta)}$ and so, $\sqrt{n}(\wh{\eta}-\eta) \todd \normal(0, A''(\eta)^{-1})$. We will see that similar results hold for distributions that are not exponential families.
\subsection{Asymptotic efficiency}
Recall that the Fisher information of a parameter $\ta$ is defined to be,
\[I(\eta) = \E_\eta[l'(\eta)^2] = -\E_\eta[l''(\eta)]. \]
We have seen that for exponential families, $I(\eta) = -\E[-A''(\eta)]=A''(\eta)$. We also have the Cramer--Rao lower bound that states if $\wh{\eta}$ is an unbiased estimator for $\eta$ based on an i.i.d. sample of size $n$, then
\[\V_\eta(\wh{\eta}) \ge \frac{1}{nI(\eta)}. \]
For exponentially families, $\V(\wh{\eta}_{MLE}) \sim \frac{1}{n A''(\eta)} = \frac{1}{nI(\eta)}$. Thus, the MLE estimator asymptotically reaches the Cramer--Rao lower bound. Estimators with this property are called \emph{asymptotically efficient}.
\begin{definition}
 An estimator $\wh{\eta}$ is \emph{asymptotically efficient} if $\V(\wh{\eta}) \sim \frac{1}{nI(\eta)}$.
\end{definition}
We can also compare the asymptotic efficiency of two estimators. 
\begin{definition}
    Let $\wh{\eta}_1$ and $\wh{\eta}_2$ be two estimators. The asymptotic relative efficiency (ARE) of $\wh{\eta}_2$ compared to $\wh{\eta}$ is,
    \[\lim_{n \to \infty} \frac{\V_\eta(\wh{\eta}_2)}{\V_\eta(\wh{\eta}_0)}. \]
\end{definition}
\subsection{The ARE of the median}
Let $X_1,X_2,\ldots$ be i.i.d. with common CDF $F$. Let $\gamma \in (0,1)$ and let $\wt{\ta}_n$ be the $[\gamma n]^{th}$ order statistic for $X_1,\ldots, X_n$, where $[y]$ denotes the ceiling of $y$. If $F(\ta) = \gamma$ and if $F'(\ta)$ exists and is strictly positive, then 
\[\sqrt{n}(\wt{\ta}_n - \ta) \todd \normal\left(0, \frac{\gamma(1-\gamma)}{[F'(\ta)]^2}\right). \]
The idea is that the event $\sqrt{n}(\wt{\ta}_n -\ta) \le a$ is equivalent to $\wt{\ta}_n \le \ta + \frac{a}{\sqrt{n}}$. Which is in turn to at least $[\gamma n]$ of $X_i$'s being less than $b=\ta + \frac{a}{\sqrt{n}}$. Thus,
\[\{\wt{\ta}_n \le a\} = \left\{\sum_{i=1}^n \one_{\{X_i \le b\}} \ge [\gamma n] \right\}. \]
Furthermore, the random variables $\one_{\{X_i \le b\}}$ are i.i.d. and with mean $F(\ta+a/\sqrt{n})$ and variance $F(\ta+a/\sqrt{n})(1-F(\ta+a/\sqrt{n}))$. Thus, we can apply the central limit theorem to and use the fact the CDF is differentiable at $\ta$. 

A special case is when $\gamma = 1/2$ and $\wt{\ta}_n$ is the median of the sample $X_1,\ldots,X_n$. For a symmetric distribution we can look at the ARE of $\wt{\ta}_n$ compared to $\bar{X}_n$. If $X_i \sim \normal(\ta,\sigma^2)$, then $\V(\bar{X}_n) = \frac{\sigma^2}{n}$ and $\V(\wt{\ta}_n) \sim \frac{1}{n}\cdot \frac{1}{4F'(\ta)}$. Note that,
\[F'(\ta) = \frac{1}{\sqrt{2\pi} \sigma}\exb{-\frac{1}{2\sigma^2}(\ta-\ta)^2} = \frac{1}{\sqrt{2\pi}\sigma^2}.\]
Thus, the ARE of the sample median compared to the sample mean is 
\[\frac{2\pi \sigma^2}{4\sigma^2} = \frac{\pi}{2} \approx 1.577 >1. \]
Thus, for Gaussian data, the sample mean has lower asymptotic variance than the sample median. 

\section{M-estimators and Z-estimators}
The MLE is defined to be,
\[\wh{\ta}_n =  \amax_{\ta \in \Ta} L(\ta|X). \]
We would like to know the following about $\wh{\ta}_n$,
\begin{enumerate}
    \item Consistency,
    \item Asymptotic distribution,
    \item Optimality.
\end{enumerate}
Since the MLE is defined as a maximizer it is an M-estimator. In many cases the MLE is also defined by solving the equation $l'(\wh{\ta})=0$, this makes $\wh{\ta}$ a Z-estimator as well. To prove consistency for M-estimators and Z-estimators, it helps to think of the log likelihood as a random function. We would like to say that the sample average of the log likelihood converges to the population log likelihood for all $\ta$. Thus, we need to understand the convergence of random functions.
\subsection{A weak law for random functions}
\begin{definition}
    Let $K$ be a compact set and let $\mu : K  \to \R$ be a continuous function. Define $\norm{\mu}_\infty = \sup_{t \in K} \abs{\mu(t)}$.
\end{definition}
\begin{lemma}
    Let $X$ be a random variable taking values in a compact set $K$. Let $h(t,x)$ be a function such that $h(\cdot, x)$ is a continuous function from $K$ to $\R$. Define $W(t) = h(t,X)$ for $t \in K$. Thus, $W$ is a random continuous function on $K$. Suppose that $\E\norm{W}_\infty < \infty$, then
    \begin{enumerate}
        \item The function $\mu(t) = \E[W(t)]$ is continuous on $K$.
        \item As $\eps \searrow 0$,
        \[ \sup_{t \in K} \E\left[\sup_{s : \norm{s-t} \le \eps} \abs{W(s)-W(t)}\right] \to 0. \]
    \end{enumerate}
\end{lemma} 
\begin{proof}
    Suppose that $t_n \to t$, then $W(t_n) \to W(t)$. Furthermore, $\abs{W(t_n)} \le \norm{W}_\infty$ and $\norm{W}_\infty$ is integrable. Thus, by the dominated convergence theorem,
    \[\mu(t_n)=\E[W(t_n)] \to \E[W(t)] = \mu(t).  \]
    Thus, $\mu$ is continuous. For each $x$ in our sample space, define
    \[M_\eps(t) = \sup_{\norm{s-t}\le \eps}\abs{W(t) - W(s)}. \]
    For fixed $t$, $M_\eps(t) \to 0$ as $\eps \to 0$ and $\abs{M_\eps(t)}\le 2\norm{W}_\infty$. Thus, by the dominated convergence theorem for each $t \in K$ we have,
    \[\lim_{\eps \to 0} \E\left[ \sup_{\norm{s-t}\le \eps}\abs{W(t) - W(s)}\right] = 0. \]
    The uniform result of the lemma is derived by the above point-wise result combined with the compactness of $K$
\end{proof}
We can use the above lemma to prove our weak law for random functions. We know from the regular weak law that $\abs{\bar{W}_n(t) - \mu(t)} \topp 0$, but we want a result that is uniform in $t$. 
\begin{theorem}
    Let $X_1,X_2,\ldots$ be i.i.d. random variables and let $W_i(t)=h(t,X_i)$ where $h$ is a function such that $h(\cdot,x):K \to \R$ is a continuous function on a compact set $K$ for all $x$. Suppose that $\E[\norm{W_1}_\infty] < \infty$. Let $\mu(t) = \E[W_1(t)]$, then
    \[\norm{W_n - \mu}_\infty \topp 0, \]
    as $n \to \infty$.  
\end{theorem}
\end{document}