\documentclass{article}
\usepackage{ae,aecompl}
\usepackage{todonotes}
\usepackage{chngcntr}
\usepackage{tikz-cd}
\usepackage{graphicx}
\graphicspath{ {./images/}}
\usepackage[all,cmtip]{xy}
\usepackage{amsmath, amscd}
\usepackage{amsthm}
\usepackage{amssymb}
\usepackage{amsfonts}
\usepackage{bm}
\usepackage{qsymbols}
\usepackage{latexsym}
\usepackage{mathrsfs}
\usepackage{mathtools}
\usepackage{cite}
\usepackage{color}
\usepackage{url}
\usepackage{enumerate}
\usepackage{verbatim}
\usepackage[draft=false, colorlinks=true]{hyperref}
\usepackage{pdfpages}
\usepackage[margin=1.2in]{geometry}
\usepackage{IEEEtrantools}
\usepackage{multirow}
\usepackage{fancyhdr}


\usepackage[nameinlink]{cleveref}


\DeclareMathOperator*{\ac}{accept}
\DeclareMathOperator*{\amax}{argmax}
\DeclareMathOperator*{\amin}{argmin}
\DeclareMathOperator*{\Aut}{Aut}
\newcommand {\al}{{\alpha}}
\newcommand {\abs}[1]{{\left\lvert#1\right\rvert}}
\newcommand {\A}{{\mathcal{A}}}
\newcommand {\AM}{{\mathrm{AM}}}
\newcommand {\AMp}{{\AM_{p}^{X}\!(\Ri_\w)}}
\newcommand {\B}{{\mathcal{B}}}
\DeclareMathOperator*{\Be}{Bern}
\newcommand {\Br}{{\dot{B}}}
\newcommand {\Ba}{{\mathfrak{B}}}
\newcommand {\C}{{\mathbb C}}
\newcommand {\ce}{\mathrm{c}}
\newcommand {\Ce}{\mathrm{C}}
\newcommand {\Cc}{\mathrm{C_{c}}}
\newcommand {\Ccinf}{\mathrm{C_{c}^{\infty}}}
\DeclareMathOperator{\DEV}{DEV}
\DeclareMathOperator{\diag}{diag}
\newcommand {\Di}{{\mathbb D}}
\newcommand {\dom}{\mathrm{dom}}
\newcommand {\ud}{\mathrm{d}}
\newcommand {\ue}{\mathrm{e}}
\newcommand {\eps}{\varepsilon}
\newcommand {\veps}{\varepsilon}
\newcommand {\vrho}{{\varrho}}
\newcommand {\E}{{\mathbb{E}}}
\newcommand {\Ec}{{\mathcal{E}}}
\newcommand {\Ell}{L}
\newcommand {\Ellp}{{L_{p}[0,1]}}
\newcommand {\Ellpprime}{{L_{p'}([0,1])}}
\newcommand {\Ellq}{{L_{q}([0,1])}}
\newcommand {\Ellqprime}{{L_{q'}([0,1])}}
\newcommand {\Ellr}{L^{r}}
\newcommand {\Ellone}{{L_{1}([0,1])}}
\newcommand {\Elltwo}{{L_{2}([0,1])}}
\newcommand {\Ellinfty}{L^{\infty}}
\newcommand {\Ellinftyc}{L_{\mathrm{c}}^{\infty}}
\newcommand {\exb}[1]{\exp\left\{#1\right\}}
\DeclareMathOperator*{\Exp}{Exp}
\DeclareMathOperator*{\Ext}{Ext}
\newcommand {\F}{{\mathcal{F}}}
\newcommand{\Fe}{{\mathbb{F}}}
\newcommand
{\G}{{\mathcal{G}}}
\newcommand {\HF}{\mathcal{H}_{\text{FIO}}^{1}(\Rd)}
\newcommand {\Hr}{H}
\newcommand {\HT}{\mathcal{H}}
\newcommand {\ui}{\mathrm{i}}
\newcommand {\I}{{I}}
\newcommand {\J}{{\mathcal{J}}}
\newcommand {\id}{{\mathrm{id}}}
\newcommand {\iid}{\stackrel{\mathclap{\normalfont\mbox{iid}}}{\sim}}
\newcommand {\im}{{\text{im }}}
\newcommand {\ind}{{\perp\!\!\!\perp}}
\newcommand{\indep}{\stackrel{\text{Indep}}{\sim}}
\DeclareMathOperator*{\Int}{int}
\newcommand {\intx}{{\overline{\int_{X}}}}
\newcommand {\inte}{{\overline{\int_{\E}}}}
\newcommand {\la}{\lambda}
\newcommand {\rb}{\rangle}
\newcommand {\lb}{{\langle}}
\newcommand {\La}{\Lambda}
\newcommand {\calL}{{\mathcal{L}}}
\newcommand {\lp}{{\mathcal{L}}^{p}}
\newcommand {\lpo}{{\overline{\mathcal{L}}^{p}\!}}
\newcommand {\Lpo}{{\overline{\Ell}^{p}\!}}
\newcommand {\M}{{\mathbf{M}}}
\newcommand {\Ma}{{\mathcal{M}}}
\newcommand {\N}{{{\mathbb N}}}
\newcommand {\Na}{{{\mathcal{N}}}}
\newcommand {\norm}[1]{\left\|#1\right\|}
\newcommand {\normm}[1]{{\left\vert\kern-0.25ex\left\vert\kern-0.25ex\left\vert #1 
    \right\vert\kern-0.25ex\right\vert\kern-0.25ex\right\vert}}
\newcommand {\Om}{{{\Omega}}}
\newcommand {\one}{{{\bf 1}}}
\newcommand {\pic}{\text{Pic }}
\newcommand {\ph}{{\varphi}}
\newcommand {\Pa}{{\mathbb{P}}}
\newcommand {\Po}{{\mathcal{P}}}
\newcommand {\Q}{{\mathbb{Q}}}
\newcommand {\R}{{\mathbb R}}
\newcommand {\Rd}{{\mathbb{R}^{d}}}
\DeclareMathOperator{\rej}{reject }
\newcommand {\Rn}{{\mathbb{R}^{n}}}
\newcommand {\cR}{{\mathcal{R}}}
\newcommand {\Rad}{{\mathrm{Rad}}}
\newcommand {\ran}{{\mathrm{ran}}}
\newcommand {\Ri}{{\mathrm{R}}}
\newcommand {\supp}{{\mathrm{supp}}}
\newcommand {\Se}{\mathrm{S}}
\newcommand {\Sp}{S^{*}(\Rn)}
\newcommand {\St}{{\mathrm{St}}}
\newcommand {\Sw}{\mathcal{S}}
\newcommand {\T}{{\mathcal{T}}}
\newcommand {\ta}{{\theta}}
\newcommand {\Ta}{{\Theta}}
\DeclareMathOperator {\V}{Var}
\newcommand {\w}{{\omega}}
\newcommand {\W}{{\mathrm{W}}}
\newcommand {\Wnp}{\text{$\mathrm{W}$\textsuperscript{$n,\!p$}}}
\newcommand {\Wnpeq}{\text{$\mathrm{W}$\textsuperscript{$n\!,\!p$}}}
\newcommand {\Wonep}{\text{$\mathrm{W}$\textsuperscript{$1,\!p$}}}
\newcommand {\Wonepeq}{\text{$\mathrm{W}$\textsuperscript{$1\!,\!p$}}}
\newcommand {\X}{{\mathcal{X}}}
\newcommand {\Z}{{{\mathbb Z}}}
\newcommand {\Za}{{\mathcal{Z}}}
\newcommand {\Zd}{{\Z[\sqrt{d}]}}
\newcommand {\vanish}[1]{\relax}

\newcommand {\wh}{\widehat}
\newcommand {\wt}{\widetilde}
\newcommand {\red}{\color{red}}

% Distributions
\newcommand{\normal}{\mathsf{N}}
\newcommand{\poi}{\mathsf{Poisson}}
\newcommand{\bern}{\mathsf{Bernoulli}}
\newcommand{\bin}{\mathsf{Binomal}}
\newcommand{\multi}{\mathsf{Multinomial}}




% put your command and environment definitions here




% some theorem environments
% remove "[theorem]" if you do not want them to use the same number sequence


  \newtheorem{thrm}{Theorem}
  \newtheorem{lemma}{Lemma}
  \newtheorem{prop}{Proposition}
  \newtheorem{cor}{Corollary}

  \newtheorem{conj}{Conjecture}
  \renewcommand{\theconj}{\Alph{conj}}  % numbered A, B, C etc

  \theoremstyle{definition}
  \newtheorem{defn}{Definition}
  \newtheorem{ex}{Example}
  \newtheorem{exs}{Examples}
  \newtheorem{question}{Question}
  \newtheorem{remark}{Remark}
  \newtheorem{notn}{Notation}
  \newtheorem{exer}{Exercise}




\title{STATS310B -- Lecture 4}
\author{Sourav Chatterjee\\ Scribed by Michael Howes}
\date{01/13/22}

\pagestyle{fancy}
\fancyhf{}
\rhead{STATS310B -- Lecture 4}
\lhead{01/13/22}
\rfoot{Page \thepage}

\begin{document}
\maketitle
\tableofcontents
\section{Stopped $\sigma$-algebras}
Recall the following definition from the previous lecture.
\begin{definition}
    Given a filtration $\{\F_n\}_{n \ge 0}$ and a stopping time $T$. The \emph{stopped $\sigma$-algebra} is defined to be
    \[\F_T = \{A \in \F : A \cap \{T=n\} \in \F_n \text{ for all } n \}.\]
\end{definition}
We saw that $\F_T$ is indeed a $\sigma$-algebra. Informally, a typical event in $\F_T$ is one that depends on $\F_n$ for $n \le T$. 
\begin{example}
    Let $S_n$ be a simple symmetric random walk on $\Z$ with $S_0=0$. Let $\F_n = \sigma(S_0,S_1,\ldots, S_n)$. We saw previously that for $a,b \in \Z$ with $a < 0 < b$, the random variable $T = \inf\{n : S_n = a \text{ or } S_n = b\}$ is a stopping time with respect to $\{F_n\}_{n \ge 0}$. We also claimed that the event $A = \{S_k \ge 0, \text{ for } k\le T\}$ was in $\F_T$. To see why this is true, take any $n$. Then,
    \begin{align*}
        A \cap \{T=n\} = \left(\bigcap_{k=0}^n \{S_k \ge 0\} \right)\cap \{T=n\}.
    \end{align*}
    Since $S_k$ is $\F_n$ measurable for $n \le k$ and since $T$ is a stopping time, both of the above events are in $\F_n$. Thus, the above intersection is in $\F_n$ and so $A \in \F_T$. 
\end{example}
\begin{proposition}
    If $S$ and $T$ are stopping times with respect to $\{F_n\}_{n \ge 0}$ and $S \le T$ always, then $\F_S \subseteq \F_T$.
\end{proposition}
\begin{proof}
    Let $A \in \F_S$ and take any $n \in \N$. Then 
    \begin{align*}
        A \cap \{T = n\} &= A \cap \{S \le n \} \cap \{T =n\}\\
        &=\left(\bigcup_{k=0}^n A \cap \{S = k\}\right) \cap \{T=n\}.
    \end{align*}
    For each $k$ we have $A \cap \{S=k\} \in \F_k \subseteq \F_n$ and so $\bigcup_{k=0}^n A \cap \{S=k\} \in \F_n$. Also, $\{T=n\} \in \F_n$ and thus $A \cap \{T=n\} \in \F_n$ and so $A \in \F_T$.
\end{proof}
Note that for the above argument, we need $S \le T$ always. The result may not be true if $S \le T$ almost surely.
\begin{example}
    Suppose that $N \in \{0,1,\ldots\}$. Then the constant random variable $T=N$ is a stopping time with respect to any filtration $\{\F_n\}_{n \ge 0}$. The stopped $\sigma$-algebra is $\F_N$. 
\end{example}
\begin{definition}
    A stopping time $T$ is \emph{bounded} if there is an integer $N$ such that $T \le N$ always.
\end{definition}
Note that if $T \le N$ always, then by proposition 1 and example 2, we have $\F_T \subseteq \F_N$. We will also define stopped random variables.
\begin{definition}
    Let $\{X_n\}_{n \ge 0}$ be a sequence of random variables adapted to $\{\F_n\}_{n \ge 0}$. Let $T$ be a stopping time with respect to $\{\F_n\}_{n \ge 0}$ such that $\Pa(T < \infty) = 1$. The random variables $X_T$ is defined on the set where $T < \infty$ by
    \[X_T(\w)=X_{T(\w)}(\w).\]
\end{definition}
Note that we can define $X_T$ on all of $\Om$ by taking $X_T(\w)$ equal to any value on the set $\{T=\infty\}$ which has probability zero. Note that the random variable $X_T$ may be very different to each of the random variables $X_n$. For example if $S_n$ is the SSRW on $\Z$ with $S_0 = 0$, then the support of $S_n$ is the set of integers in $\{-n,-n+1,\ldots, n-1,n\}$ with the same parity as $n$. However, if $T = \inf\{n : S_n = a \text{ or } S_n =b\}$, then $S_T \in \{a,b\}$ almost surely (we will see shortly that $\Pa(T < \infty))=1$. 

We will now state a useful theorem.

\begin{theorem}[Optional stopping theorem]
    Let $\{X_n\}_{n \ge 0}$ be a martingale adapted to a filtration $\{\F_n\}_{n \ge 0}$. Let $S$ and $T$ be bounded stopping times with respect to $\{F_n\}$ such that $S \le T$ always. Then $X_S$ and $X_T$ are integrable and $\E(X_T|\F_S) = X_S$ almost surely. In particular $\E[X_T]=\E[X_0]$.
\end{theorem}
Note that when $T$ and $S$ are both constant this reduces to the result that for all $n,k$ such that $k \le n$,
\[\E(X_n|\F_k) =X_k ~~\text{ and }~~\E[X_n]=\E[X_0].\]
The above results can be proved directly by using induction and the tower property. We will now prove the optional stopping theorem.

\begin{proof}
    Let $N$ be an integer such that $S \le T \le N$ always. Note that
    \[\abs{X_S},\abs{X_T} \le \abs{X_0}+\abs{X_1}+\ldots+\abs{X_N}. \]
    Thus, $X_S$ and $X_T$ are both integrable. We will next show that $X_S$ is $\F_S$ measurable. Fix any $c \in \R$, we wish to show that $\{X_S \le c \} \in \F_S$ which requires $\{X_S \le c\} \cap\{S =n\} \in \F_n$ for every $n$.
    \begin{align*}
        \{X_S \le c\} \cap \{S = n\} &= \{X_n \le c\} \cap\{ S=n\}.
    \end{align*}
    Both $\{S=n\}$ and $\{X_n \le c\}$ are in $\F_n$ and so $\{X_S \le c\} \in \F_S$. Now let $A \in \F_S$. We wish to show
    \[\E[X_T \one_A] = \E[X_S \one_A]. \]
    Note that
    \begin{align*}
        \E[X_N \one_A]&=\E\left[X_N \sum_{n=0}^N \one_{\{T=n\}}\one_A\right]\\
        &=\sum_{n=0}^N \E[X_N \one_{\{T=n\}}\one_A]\\
        &=\sum_{n=0}^N \E[X_N \one_{\{T=n\}\cap A}].
    \end{align*}
    We know that $\{T=n\} \cap A \in \F_n$ since $A \in \F_S \subseteq \F_T$. Thus 
    \begin{align*}
        \sum_{n=0}^N \E[X_N \one_{\{T=n\}\cap A}]&=\sum_{n=0}^N \E[\E(X_N|\F_n) \one_{\{T=n\}\cap A}]\\
        &=\sum_{n=0}^N \E[X_n \one_{\{T=n\}\cap A}]\\
        &=\E\left[\sum_{n=0}^N X_n \one_{\{T=n\}}  \one_A\right]\\
        &=\E[X_T\one_A].
    \end{align*}
    By the same argument we have $\E[X_N \one_A] = \E[X_S \one_A]$. Thus, $\E[X_T \one_A]=\E[X_S \one_A]$ and thus 
    \[\E(X_T|\F_S)= X_S. \]
    If we take $S=0$, then we get 
    \[\E[X_T] = \E[\E(X_T|\F_0)]=\E[X_0].\qedhere  \]
\end{proof}
The requirement that $S$ and $T$ are bounded stopping times may seem restrictive, but the result is not true if we do not have this assumption. Fortunately we can often approximate an unbounded stopping time by a sequence of stopping times.
\begin{proposition}
    Let $T$ be a stopping time with respect to a filtration $\{F_k\}_{k \ge 0}$. Then for every $n \in \N$, the random variable
    \[T \land n = \min\{T,n\}, \]
    is a stopping time bounded by $n$. Furthermore, $T \land n \to T$ on the set where $T < \infty$. 
\end{proposition}
\begin{proposition}
    To see that $T \land n$ is a stopping time, note that
    \[\{T\land n = k\} =\begin{cases}
        \{T=k\} & \text{if } k < n,\\
        \{T \ge k\} & \text{if } k = n,\\
        \emptyset & \text{if } k  > n.
    \end{cases} \]
    Thus, we immediately see that $\{T \land n = k \} \in \F_k$ when $k \neq n$. When $k = n$, note that
    \[\{T \ge k \} = \{T < k\}^C = \left(\bigcup_{j=1}^{k-1}\{T =j\}\right)^C. \]
    We know that $\{T=j\} \in \F_j \subseteq \F_k$ for all $j < k$ and so $\{T \ge k \} \in \F_k$. 
\end{proposition}
\section{Gambler's ruin}
We will now study an example which shows the usefulness of the optional stopping theorem. Let $\{S_n\}_{n \ge 0}$ be a SSRW with $S_0 = 0$ and as before let $T=\inf \{n : S_n = a \text{ or } S_n = b\}$. We are interested in 
\[\Pa(T < \infty, S_T = b). \]
We will soon see that $\Pa(T < \infty) =1$. Thus, the probability we are interested in is
\[\Pa(S_T = b). \]
\begin{remark}
    This question is related to the \emph{gambler's ruin}. Consider a gambler with $x \in \{1,2,\ldots\}$ dollars. The gambler repeatedly makes bets where they can either win or lose $\$1$. They continue playing until they make $y > x$ dollars, or they go broke. The probability that they don't go broke is equal to the probability that they end with $y$ dollars. This is equal to $\Pa(S_T = b)$ when $a = -x$ and $b = y-x$.
\end{remark}
Note that $S_T$ only takes the values $a$ and $b$. Thus,
\[\E[S_T] = a\Pa(S_T=a)+b\Pa(S_T=b)=a(1-\Pa(S_T=b)) + b\Pa(S_T=b)= a+(b-a)\Pa(S_T=b). \]
Thus, if we can calculate $\E[S_T]$, then we will know $\Pa(S_T=b)$ since 
\[\Pa(S_T=b) = \frac{\E[S_T]-a}{b-a}.\]
We wish to use the optional stopping theorem to conclude that $\E[S_T]=\E[S_0]=0$. The stopping time $T \land n$ is bounded for each $n$, and so we know that $\E[S_{T \land n}] = 0$. We also have $\abs{S_{T \land n}}\le \max\{-a,b\}$. Thus, it suffices to show that $T\land n \to T$ almost surely. To do this, we need to show that $\Pa(T < \infty) = 1$.
\begin{proposition}
    With $S_n$ and $T$ as above, $\Pa(T < \infty) =1$.
\end{proposition}
\begin{proof}
    We can divide the set $\{1,2,3,\ldots\}$ into infinitely many blocks $B_j$ of consecutive integers such that $B_j$ contains $\abs{a}+b$ integers. Let $A_j$ be the event that $S_k-S_{k-1} = 1$ for all $k \in B_j$. Each $A_j$ has probability $\Pa(A_j) = 2^{-\abs{a}-b}$ and the events $A_j$ are independent since the sets $B_j$ are disjoint. Thus, by the second Borel--Cantelli lemma,
    \[\Pa(A_j, \text{ infinitely often}) =1. \]
    For each $A_j$ we have $A_j \subseteq \{T < \infty\}.$ To see this let $k_1$ be the minimum of $B_j$. Suppose that $\w \in A_j$. If $T(\w) > k_1$, then $S_{k_1}(\w) \ge a$. Since $\w \in A_j$, 
    \[S_{k_1+\abs{a}+b}(\w) = S_{k_1}(\w) + \abs{a}+b = b.\] 
    Thus, $T(\w) \le k_1+\abs{a}+b < \infty$. We thus have $A_j \subseteq \{T < \infty\}$, and so
     \[\Pa(T < \infty)\ge \Pa(A_j, \text{ infinitely often})=1.\] 
\end{proof}
Thus, $T\land n \to T$ almost surely and furthermore more $S_{T\land n} \to S_T$. Since $S_{T\land n}$ is uniformly bounded we have
\[\E[S_T] =\lim_{n \to \infty} \E[S_{T\land n}] = \lim_{n\to \infty}\E[S_0] =0. \]
We thus have 
\[\Pa(S_T = b) = \frac{-a}{-a+b}. \]
In terms of the gambler's ruin this equal $\frac{x}{y}$. So if the gambler starts with $x=900$ dollars, they have a $90\%$ of making $y=1000$ dollars before going broke. But as we see, the gambler will have to be very patient. We can also use the optional stopping theorem to calculate the expected value of $T$ as well. Recall that 
\[M_n = S_n^2 - n,\]
is a martingale since $\V(X_n)=1$. And thus 
\[\E[M_{T \land n}] = \E[S_{T \land n}^2] - \E[T \land n]. \]
The random variables $S_{T \land n}^2$ are uniformly bounded by $\max\{a^2,b^2\}$ and converge almost surely to $S_T^2$. Furthermore, $T \land n$ are all non-negative, and they converge almost surely to $T$. Therefore, by the dominated and monotone convergence theorem, we have
\[\E[S_{T}^2] = \lim_{n \to \infty} \E[S_{T \land n}^2] = \lim_{n \to \infty}\E[T\land n] = \E[T]. \]
We also know that
\begin{align*}
    \E[S_T^2]&=a^2\Pa(S_T = a) + b^2\Pa(S_T = b)\\
    &=\frac{a^2b}{b-a} + \frac{-b^2a}{b-a}\\
    &= \frac{ab(a-b)}{b-a}\\
    &=-ab.
\end{align*}
In the gambler's ruin, this equals $x(y-x)$. So to get from $\$900$ to $\$1000$ or go broke, the gambler will have to wait on average $90000$ turns.
\section{Sub-martingales and super-martingales}
We will now state two definitions that generalize martingales.
\begin{definition}
    Let $\{F_n\}_{n \ge 0}$ be a filtration and let $\{X_n\}_{n \ge 0}$ be a sequence of adapted, integrable random variables. Then,
    \begin{enumerate}
        \item $\{X_n\}_{n \ge 0}$ is a \emph{sub-martingale} if for every $n$, \[\E(X_{n+1}|\F_n) \ge X_n.~\text{a.s}\]
        \item $\{X_n\}_{n \ge 0}$ is a \emph{super-martingal}e if for every $n$, \[\E(X_{n+1}|\F_n) \le X_n.~\text{a.s}\]
    \end{enumerate}
\end{definition}
\begin{remark}
    One can use Jensen's inequality to make sub-martingales out of martingales. Let $\{M_n\}_{n \ge 0}$ be a martingale and let $\phi$ be a convex function such that $X_n = \phi(M_n)$ is integrable for every $n \ge 0$. Then $\{X_n\}_{n \ge 0}$ is a sub-martingale since, almost surely
\end{remark}
\end{document}